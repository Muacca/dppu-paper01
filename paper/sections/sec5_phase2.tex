\section{Phase 2: Nieh--Yan boundary term and the precession mode}
\label{sec:phase2}

In Phase~1 (Sec.~\ref{sec:phase1}) we showed that, within the Spinning
$SU(2)$ handle ansatz, pure TEGR admits a classically stable equal--radius
configuration $r(\psi)=r_0$ whose energy scales as
$E_{\text{TEGR}}(q)\propto |q|$ for fixed non-zero twist number $m$.
This background is parity--even: it does not distinguish between $q$ and
$-q$ and does not select any preferred handedness for the handle.

In this section we switch on the Nieh--Yan term and introduce a slow
precession mode that tilts the handle axis by a small time-dependent
angle $\varepsilon(t)$.  We show that the combined effect of TEGR and
Nieh--Yan is an effective potential of the form
\begin{equation}
  V_{\text{eff}}(\varepsilon; q)
  \;\simeq\;
  E_{\text{TEGR}}(q)
  + \frac{1}{2}k(q)\,\varepsilon^2
  + \Lambda_q\,\varepsilon
  + \mathcal{O}(\varepsilon^4),
  \label{eq:Veff-intro}
\end{equation}
where $k(q)>0$ is a positive function of $q$ and $\Lambda_q$ is a
coefficient linear in $q$.  In the sign convention adopted here we write
\begin{equation}
  \Lambda_q \;=\; -\,\gamma\,q,
  \qquad \gamma>0 \;\;\text{(for fixed $m,\omega,\theta_{\text{NY}},\Delta r^2$)},
  \label{eq:Lambdaq-sign}
\end{equation}
so that $\Lambda_{-q}=-\Lambda_q$ and the sign of $\Lambda_q$ is opposite
to the sign of $q$.

The minimum of \eqref{eq:Veff-intro} occurs at
\begin{equation}
  \varepsilon_*(q)
  \;\simeq\;
  -\,\frac{\Lambda_q}{k(q)}
  \;=\;
  \frac{\gamma}{k(q)}\,q,
  \label{eq:epsilon-star-intro}
\end{equation}
so that, for $\Lambda_q\neq 0$, the true ground state is slightly tilted
($\varepsilon_*\neq 0$) and its sign is tied to the sign of $q$.  This
provides a simple geometric mechanism for dynamical chirality selection:
handles with $q>0$ and $q<0$ precess in opposite directions.

Throughout this section we work in the regime
\begin{equation}
  |\varepsilon(t)| \ll 1,
  \qquad
  \Bigl|\frac{d\varepsilon}{dt}\Bigr|
  \;\text{small},
\end{equation}
so that an expansion in powers of $\varepsilon$ and its time derivatives
is consistent.  Large precession angles require a separate non-perturbative
treatment and lie beyond the scope of the present paper.

%------------------------------------------------------------
\subsection{Introducing the precession mode}
\label{subsec:phase2-precession}

We start from the equal--radius background of Phase~1,
\begin{equation}
  r(\psi) = r_0,
\end{equation}
with the Spinning $SU(2)$ handle ansatz of Sec.~\ref{sec:ansatz},
schematically
\begin{equation}
  U(\psi,t)
  \;=\;
  \text{Twist}(m\psi)\,\text{Spin}(\omega t),
\end{equation}
and the associated tetrad $e^a{}_\mu$ that realises a wormhole--like handle
of radius $r_0$ and twist number $m$.

To describe a slow precession of the handle axis, we introduce a small,
spatially uniform rotation in the internal $1$--$2$ plane,
\begin{equation}
  e^1 + i e^2
  \;\longrightarrow\;
  e^{i\varepsilon(t)} (e^1 + i e^2),
  \qquad
  |\varepsilon(t)| \ll 1,
  \label{eq:precession-rotation}
\end{equation}
while keeping $r(\psi)=r_0$ fixed.
Concretely, at the level of the $SU(2)$ field this precession mode can
be implemented by an additional factor
\begin{equation}
  U_{\text{prec}}(t)
  \;=\;
  \exp\!\left(\frac{i}{2}\,\varepsilon(t)\,\sigma_2\right),
  \label{eq:U-prec}
\end{equation}
acting on the left of the background configuration, so that the full
$SU(2)$ field becomes
\begin{equation}
  U(\psi,t)
  \;=\;
  U_{\text{prec}}(t)\,
  \exp\!\left(i\frac{m\psi}{2}\sigma_3\right)
  \exp\!\left(i\frac{\omega t}{2}\sigma_1\right).
  \label{eq:U-with-prec}
\end{equation}
Via the $SU(2)\to SO(3)$ map, this corresponds to a time--dependent
rotation of the spatial triad about the internal 2--axis, i.e. a slow
tilt of the handle axis towards the $e^2$--direction.

We treat $\varepsilon(t)$ as a collective coordinate parametrising the
slow precession of the entire handle.  Inserting the precessing tetrad
into the TEGR action and expanding around $\varepsilon=0$, one obtains an
effective contribution to the energy of the form
\begin{equation}
  E_{\text{TEGR}}(q,\varepsilon)
  \;=\;
  E_{\text{TEGR}}(q)
  + \frac{1}{2}k(q)\,\varepsilon^2
  + \mathcal{O}(\varepsilon^4),
  \label{eq:V-TEGR-eps}
\end{equation}
where:

\begin{itemize}
  \item $E_{\text{TEGR}}(q)$ is the equal--radius energy derived in
        Sec.~\ref{sec:phase1}, scaling as $\propto |q|$ for fixed $m$;
  \item $k(q)>0$ is an effective \emph{tilt stiffness} generated by the
        TEGR sector.
\end{itemize}

Because the TEGR action is parity--even and invariant under
$\varepsilon \to -\varepsilon$, only even powers of $\varepsilon$ appear,
and the leading nontrivial term is quadratic.  The positivity of $k(q)$
ensures that, in the absence of the Nieh--Yan term, the configuration
$\varepsilon=0$ is a local minimum.

At this stage we do not need the explicit expression for $k(q)$.  We only
assume that $k(q)$ is positive for the range of $q$ of interest.  In
Phase~3 we will show that, within the present ansatz, $k(q)$ scales
linearly with $|q|$,
\begin{equation}
  k(q) \;\propto\; |q|,
\end{equation}
which will play a key rôle in balancing the TEGR energy against the
Nieh--Yan contribution.

%------------------------------------------------------------
\subsection{Nieh--Yan term as an effective boundary coupling}
\label{subsec:phase2-NY}

In Sec.~\ref{sec:TEGR} we introduced the Nieh--Yan 4--form
\begin{equation}
  \mathcal{N}
  \;=\;
  d\bigl(e^a \wedge T_a\bigr)
  \;=\;
  T^a \wedge T_a - e^a \wedge e^b \wedge R_{ab}[\omega],
\end{equation}
and the corresponding topological term
\begin{equation}
  S_{\text{NY}}
  \;=\;
  \theta_{\text{NY}}
  \int_M \mathcal{N},
  \label{eq:S-NY-def}
\end{equation}
with dimensionless coupling $\theta_{\text{NY}}$.  In the teleparallel
setting $R_{ab}[\omega]=0$, so that $\mathcal{N}$ is exact,
\begin{equation}
  \mathcal{N}
  \;=\;
  d\bigl(e^a \wedge T_a\bigr),
\end{equation}
and on a manifold with (effective) boundary $\partial M$ one has
\begin{equation}
  S_{\text{NY}}
  \;=\;
  \theta_{\text{NY}}
  \int_{\partial M} e^a \wedge T_a.
  \label{eq:S-NY-boundary}
\end{equation}

For the Spinning $SU(2)$ handle we take
\begin{equation}
  M \simeq \mathbb{R}_t \times S^1_\psi \times S^2
\end{equation}
and model the two ends of the handle in the $\psi$--direction (and/or the
junctions to the surrounding bulk) as an effective boundary $\partial M$.
The detailed matching to the bulk geometry is not needed here; all that
matters is that the precessing tetrad induces a nontrivial value of
$e^a \wedge T_a$ on $\partial M$.

For the non–precessing Spinning $SU(2)$ handle, Sec.~\ref{sec:ansatz}
showed that the Nieh--Yan density takes, up to an overall normalisation,
the exact total–derivative form
\begin{equation}
  \mathcal{N}
  \;\simeq\;
  q\,\omega\,m\,
  d\!\bigl(r(\psi)^2\bigr)\wedge dt\wedge d\Omega_2,
  \label{eq:NY-total-derivative-phase2}
\end{equation}
where $d\Omega_2$ is the area form on the unit 2–sphere.  Integrating
over $\psi$ and $S^2$ one obtains a contribution proportional to the
difference of $r^2$ at the two endpoints of the handle,
\begin{equation}
  \int_M \mathcal{N}
  \;\propto\;
  q\,\omega\,m\,
  \bigl[r(\psi)^2\bigr]_{\psi\text{--endpoints}}.
  \label{eq:NY-boundary-phase2}
\end{equation}

The precession mode $\varepsilon(t)$ modifies the matching between the
handle and the bulk at the endpoints and effectively shifts the boundary
values of $r^2$ by an amount proportional to $\varepsilon(t)$:
schematically,
\begin{equation}
  \Delta\!\bigl(r^2\bigr)
  \;\longrightarrow\;
  \Delta\!\bigl(r^2\bigr)
  + \beta\,\varepsilon(t),
\end{equation}
with some positive constant $\beta$ determined by the geometry of the
junction.  Inserting this into \eqref{eq:NY-boundary-phase2} and then
into \eqref{eq:S-NY-def} yields an effective contribution to the action
of the form
\begin{equation}
  S_{\text{NY}}[\varepsilon]
  \;=\;
  \int dt\; L_{\text{NY}}(\varepsilon)
  \;\simeq\;
  \int dt\; \Lambda_q\,\varepsilon(t)
  + \mathcal{O}(\varepsilon^3),
  \label{eq:S-NY-eps}
\end{equation}
where the coefficient $\Lambda_q$ is a constant (for a fixed handle)
given schematically by
\begin{equation}
  \Lambda_q
  \;=\;
  -\,\theta_{\text{NY}}\,
  C_{\text{NY}}\,
  q\, m\, \omega\, \Delta r^2,
  \label{eq:Lambda-q-def}
\end{equation}
with $C_{\text{NY}}>0$ encoding the detailed tetrad normalisation and
boundary geometry.  The minus sign in \eqref{eq:Lambda-q-def} matches
the sign convention adopted in \eqref{eq:Lambdaq-sign}.  The precise
derivation of \eqref{eq:Lambda-q-def} and the identification of the positive constant
$C_{\text{NY}}$ are deferred to Appendix~D; for the discussion in Phase~2 we
only need the following qualitative properties:
\begin{itemize}
  \item $\Lambda_q$ is \emph{linear} in the monopole charge $q$,
        with $\Lambda_{-q} = -\Lambda_q$;
  \item $\Lambda_q$ vanishes if either $q=0$ or $\theta_{\text{NY}}=0$;
  \item for fixed $m,\omega,\Delta r^2$ and $\theta_{\text{NY}}$,
        the sign of $\Lambda_q$ is opposite to that of $q$.
\end{itemize}

In the static limit, $S_{\text{NY}}$ contributes to the energy as
\begin{equation}
  V_{\text{NY}}(\varepsilon; q)
  \;=\;
  \Lambda_q\,\varepsilon
  + \mathcal{O}(\varepsilon^3),
  \label{eq:V-NY-eps}
\end{equation}
which is \emph{odd} in $\varepsilon$ and changes sign under
$q \to -q$.

Physically, \eqref{eq:V-NY-eps} plays the rôle of an ``external field''
for the precession mode: it lifts the degeneracy between $\varepsilon>0$
and $\varepsilon<0$ that is present in the TEGR sector.

%------------------------------------------------------------
\subsection{Effective potential and chirality selection}
\label{subsec:phase2-chirality}

Combining the TEGR and Nieh--Yan contributions
\eqref{eq:V-TEGR-eps} and \eqref{eq:V-NY-eps}, we obtain the effective
static potential for the precession mode,
\begin{equation}
  V_{\text{eff}}(\varepsilon; q)
  \;=\;
  E_{\text{TEGR}}(q)
  + \frac{1}{2}k(q)\,\varepsilon^2
  + \Lambda_q\,\varepsilon
  + \mathcal{O}(\varepsilon^4).
  \label{eq:V-eff}
\end{equation}
Minimising with respect to $\varepsilon$ yields
\begin{equation}
  \frac{\partial V_{\text{eff}}}{\partial \varepsilon}
  \;=\;
  k(q)\,\varepsilon + \Lambda_q
  + \mathcal{O}(\varepsilon^3)
  \;=\; 0,
\end{equation}
so that, in the small--$\varepsilon$ regime,
\begin{equation}
  \varepsilon_*(q)
  \;\simeq\;
  -\,\frac{\Lambda_q}{k(q)},
  \qquad
  |\varepsilon_*(q)| \ll 1.
  \label{eq:epsilon-star}
\end{equation}
Using \eqref{eq:Lambdaq-sign}, this may be written as
\begin{equation}
  \varepsilon_*(q)
  \;\simeq\;
  \frac{\gamma}{k(q)}\,q.
\end{equation}
The corresponding shift of the vacuum energy is
\begin{equation}
  V_{\text{eff}}(\varepsilon_*; q)
  \;\simeq\;
  E_{\text{TEGR}}(q)
  - \frac{\Lambda_q^2}{2k(q)},
\end{equation}
so that the Nieh--Yan term lowers the energy by an amount
$\propto \Lambda_q^2/k(q)$ whenever $\Lambda_q\neq 0$.

From \eqref{eq:Lambdaq-sign} it follows that
\begin{equation}
  \varepsilon_*(-q)
  \;\simeq\;
  -\,\varepsilon_*(q),
\end{equation}
for fixed $m,\omega,\theta_{\text{NY}},\Delta r^2$, because
$\Lambda_{-q} = -\Lambda_q$ and $k(q)$ is positive and depends only on
$|q|$.  Thus the sign of the equilibrium tilt is tied to the sign of
the monopole charge: for one sign of $q$ the handle prefers a slightly
``left--tilted'' configuration, while for the opposite sign it prefers
a ``right--tilted'' one.

It is convenient to interpret the sign of $\varepsilon$ as a geometric
proxy for chirality.  In this language, Phase~2 shows that:
\begin{itemize}
  \item in pure TEGR ($\Lambda_q=0$) the two chiralities $\varepsilon>0$
        and $\varepsilon<0$ are exactly degenerate and the vacuum is
        parity--even;
  \item once the Nieh--Yan term is included with nonzero
        $\theta_{\text{NY}}$, the degeneracy is lifted and the ground
        state develops a small chiral bias
        $\varepsilon_*(q)\propto q$;
  \item the direction of this bias (which chirality is favoured) is
        controlled by the sign of $\theta_{\text{NY}}$ and by the
        signs of $m$ and $\omega$ encoded in $\Lambda_q$.
\end{itemize}

In particular, for $q=0$ or $\theta_{\text{NY}}=0$ one has
$\Lambda_q = 0$ and the effective potential reduces to a purely even
function of $\varepsilon$,
\begin{equation}
  V_{\text{eff}}(\varepsilon; q=0)
  \;\simeq\;
  E_{\text{TEGR}}(0)
  + \frac{1}{2}k(0)\,\varepsilon^2
  + \mathcal{O}(\varepsilon^4),
\end{equation}
so that the ground state returns to $\varepsilon_*=0$ and parity is
preserved.  Nonzero $q$ and nonzero $\theta_{\text{NY}}$ are therefore
both necessary ingredients for dynamical chirality selection in this
framework.

The observed unique chirality of weak interactions would, in this
picture, require a cosmological mechanism that selects one sign of
$\theta_{\text{NY}}$ (and hence one global sign of $\Lambda_q$) across
the observable universe.  A concrete realisation of such a mechanism is
left for future work.

%------------------------------------------------------------
\subsection{Range of validity and link to Phase 3}
\label{subsec:phase2-range}

The analysis in this section relies on several approximations:

\begin{itemize}
  \item The precession angle is small,
        $|\varepsilon|\ll 1$, and higher powers
        $\mathcal{O}(\varepsilon^3)$ are neglected.
  \item The precession is spatially uniform along $S^1_\psi$ and $S^2$,
        so that only the zero mode of $\varepsilon(t)$ is retained.
  \item The backreaction of $\varepsilon$ on the radius $r(\psi)$ and on
        the torsion flux $q$ is neglected at leading order.
\end{itemize}

Within this regime, Eq.~\eqref{eq:V-eff} provides a controlled
description of how the Nieh--Yan term biases the precession mode and
induces a small chiral tilt proportional to $q$.  The TEGR part of the
potential is encoded in the stiffness $k(q)$, whose detailed $q$--dependence
is not yet fixed in this section.

In Phase~3 we will analyse $k(q)$ more systematically within the same
Spinning $SU(2)$ handle ansatz, and show that its scaling with $|q|$
leads to a near--critical balance between the TEGR energy $\propto |q|$
and the Nieh--Yan induced energy gain.  This sets the stage for
discussing multi--handle configurations and possible composite states.
