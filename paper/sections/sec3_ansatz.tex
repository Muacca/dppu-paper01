\section{Spinning $SU(2)$ Handle Ansatz}
\label{sec:ansatz}

In this section we specify the microscopic geometry of a single handle and
introduce the \emph{Spinning $SU(2)$ Handle Ansatz} that will be used
throughout Phases~1--3. The ansatz encodes three charges $(q,m,\omega)$
and a radius profile $r(\psi)$ in a tetrad field $e^a{}_\mu$ defined on a
fixed handle topology.

\subsection{Geometry of a single handle}
\label{subsec:ansatz-geometry}

Locally we model a single microscopic handle as a product manifold
\begin{equation}
  M \simeq \mathbb{R}_t \times S^1_\psi \times S^2_{(\theta,\phi)},
\end{equation}
with coordinates
\begin{equation}
  x^\mu = (t,\psi,\theta,\phi),
\end{equation}
where $t$ denotes physical time, $\psi\in[0,2\pi)$ is an angular coordinate
along the handle axis ($S^1$), and $(\theta,\phi)$ are polar and azimuthal
angles on the cross--section $S^2$.

We assume that the handle is much smaller than any macroscopic curvature
scale of the ambient universe, so that background fields may be treated
as approximately constant over $M$.

As in Phase~1, we keep a single geometric degree of freedom for the
shape of the handle: the physical radius $r(\psi)$ of the two--sphere
cross--section as a function of the axial coordinate~$\psi$.
A rough form of the line element is then
\begin{equation}
  ds^2 \sim -dt^2 + d\psi^2
            + r(\psi)^2\bigl(d\theta^2 + \sin^2\theta\,d\phi^2\bigr),
  \label{eq:ansatz-metric-rough}
\end{equation}
while the exact expression will be encoded in the tetrad defined below.
In particular, the $q$--dependent terms that implement the torsion flux
are captured at the tetrad level and are not shown explicitly in
\eqref{eq:ansatz-metric-rough}.

\subsection{Tetrad construction and parameters}
\label{subsec:ansatz-tetrad}

We work in the Weitzenb\"ock gauge, so that the spin connection vanishes
and all gravitational information is carried by the tetrad field
$e^a{}_\mu$.

The handle ansatz is constructed in two steps:
we first choose a convenient \emph{reference tetrad} $\tilde e^a{}_\mu$
with a monopole--like torsion flux, and then act with
a time-- and angle--dependent spatial rotation derived from an
$SU(2)$ group element.

\subsubsection*{Reference tetrad and monopole charge $q$}

For definiteness we take the reference tetrad one--forms to be
\begin{align}
  \tilde e^0 &= dt, \label{eq:ref-tetrad-0}\\
  \tilde e^1 &= d\psi, \label{eq:ref-tetrad-1}\\
  \tilde e^2 &= r(\psi)\,d\theta, \label{eq:ref-tetrad-2}\\
  \tilde e^3 &= r(\psi)\,\sin\theta\,
               \bigl(d\phi + q(1-\cos\theta)\,d\psi\bigr).
               \label{eq:ref-tetrad-3}
\end{align}
The resulting metric is approximately of the form
\eqref{eq:ansatz-metric-rough}; the $q$--dependent mixing term resides
in $\tilde e^3$ and will play a role only in the torsion sector.

We define $q\in\mathbb{Z}$ as an \emph{integer monopole charge} associated
with the torsion flux through each cross--section $S^2_\psi$ at fixed
$(t,\psi)$.
Concretely, one may impose a normalisation condition of the form
\begin{equation}
  \frac{1}{4\pi}\int_{S^2_\psi} \!\!\!\star T^1 = q,
\end{equation}
where $T^a = d\tilde e^a$ is the torsion 2--form and $\star$ denotes the
Hodge dual built from the induced metric.
In what follows we treat $q$ as a fixed topological input for each handle.

\subsubsection*{$SU(2)$ field and induced rotation}

We introduce an internal $SU(2)$ field
\begin{equation}
  U(t,\psi)
  \;=\;
  \exp\!\left(i\frac{\omega t}{2}\sigma_1\right)\,
  \exp\!\left(i\frac{m\psi}{2}\sigma_3\right),
  \label{eq:SU2-U}
\end{equation}
where $\sigma_i$ are the Pauli matrices.
The first factor implements an intrinsic \emph{spin} in time, while the
second factor implements a \emph{twist} along the handle axis.
We therefore refer to \eqref{eq:SU2-U} as a ``spin--then--twist''
parameterisation.

Via the double cover $SU(2)\to SO(3)$, the field $U(t,\psi)$ defines
a spatial rotation matrix $\Lambda^i{}_j(t,\psi)$ acting on the spatial
tetrad indices $i,j=1,2,3$.
We then define the physical tetrad as
\begin{equation}
  e^0 = \tilde e^0,
  \qquad
  e^i = \Lambda^i{}_j(t,\psi)\,\tilde e^j,
  \quad i=1,2,3.
  \label{eq:phys-tetrad}
\end{equation}
In the explicit implementation used in the detailed calculations,
the rotation factorises as
\begin{equation}
  \Lambda(t,\psi) = R_3(m\psi)\,R_1(\omega t),
  \label{eq:rot-matrix-factor}
\end{equation}
so that the spatial basis is first spun around the local
$\tilde e^1$--axis as time evolves, and then twisted in the
$(\tilde e^2,\tilde e^3)$--plane as one moves along $\psi$.

A key property of this ordering is that the handle axis vector $e^1$
remains \emph{time--independent}: the spin acts around the axis rather
than precessing the axis itself.
This will allow us to add a small precession mode $\varepsilon(t)$
in Phase~2 without destabilising the background configuration.

\subsubsection*{Summary of parameters}

The ansatz introduces the following parameters and fields:
\begin{itemize}
  \item $q\in\mathbb{Z}$ (flux / monopole charge):
    integer--quantised torsion flux through the $S^2$ cross--section.
    It controls the strength of the core repulsion and, via the
    Nieh--Yan/WZW chain, the effective WZW level in a boundary
    Skyrme--like description.
  \item $m\in\mathbb{Z}$ (twist number):
    twist number along the handle axis; as $\psi$ increases from $0$
    to $2\pi$, the internal frame rotates $m$ times in the cross--section.
  \item $\omega\in\mathbb{R}$ (spin):
    intrinsic spin angular frequency of the handle, defined in the
    co--moving frame of the twisted handle.
  \item $r(\psi) > 0$ (radius profile):
    radius of the $S^2$ cross--section as a function of $\psi$.
    In Phase~1 we focus on the equal--radius ansatz $r(\psi)=r_0$,
    while Phases~2 and~3 allow small deformations.
\end{itemize}

In the TEGR action, $q$, $m$, $\omega$ and $r(\psi)$ control,
respectively, the flux--induced core energy, the twist energy,
the spin--related terms, and the elastic energy associated with
variations of the radius.
Their scaling behaviours will be analysed in detail in
Sec.~\ref{sec:phase1} and subsequent sections.

\subsection{Nieh--Yan density as an exact form}
\label{subsec:ansatz-NY}

In the Weitzenb\"ock geometry adopted here, the Nieh--Yan 4--form
simplifies to
\begin{equation}
  \mathcal N = T^a\wedge T_a = d(e^a\wedge T_a),
  \label{eq:NY-exact}
\end{equation}
and behaves as a torsion--built topological invariant.

For the Spinning $SU(2)$ Handle Ansatz, an explicit calculation of the
torsion 2--forms $T^a = de^a$ yields a Nieh--Yan density that, up to an
overall normalisation fixed by the choice of $q$, takes the form
\begin{align}
  \mathcal N
  &\simeq
  q\,\omega\,m\,
  \partial_\psi\bigl(r(\psi)^2\bigr)\,
  dt\wedge d\psi\wedge
  \sin\theta\,d\theta\wedge d\phi
  \nonumber\\
  &=\;
  q\,\omega\,m\,
  d\!\bigl(r(\psi)^2\bigr)\wedge dt\wedge d\Omega_2,
  \label{eq:NY-total-derivative}
\end{align}
where $d\Omega_2 = \sin\theta\,d\theta\wedge d\phi$ is the area form
on the unit $2$--sphere, and $\simeq$ denotes equality up to a
numerical normalisation that will be fixed once and for all by the
flux quantisation condition for $q$.

This total--derivative structure has two important consequences.

\paragraph{(i) Boundary sensitivity.}
Because $\mathcal N$ is an exact form in the $\psi$--direction,
the Nieh--Yan contribution of a single handle depends only on the
boundary values of $r^2$ at the ends of the handle (or, more precisely,
at the joints where the handle is glued to the ambient manifold):
\begin{equation}
  \int_M \mathcal N
  \;\propto\;
  q\,\omega\,m\,
  \bigl[r(\psi)^2\bigr]_{\psi\text{--endpoints}}.
  \label{eq:NY-boundary}
\end{equation}
For a microscopic handle that is nearly cylindrical in its interior,
the contribution is thus localised near the joints.

\paragraph{(ii) Effective WZW level.}
We emphasise that $q$ is defined independently, as the integer monopole
charge of the torsion flux in the reference tetrad, and is \emph{not}
redefined via the Nieh--Yan integral.

Instead, we interpret the Nieh--Yan integral as a measure of an
effective WZW--like level seen by a boundary Skyrme field.
For a single handle, the scaling behaviour implied by
\eqref{eq:NY-total-derivative} can be summarised as
\begin{equation}
  \int_M \mathcal N
  \;\propto\;
  q\,\omega\,m\,
  \Delta\!\bigl(r^2\bigr),
  \label{eq:NY-WZW-scaling}
\end{equation}
where $\Delta(r^2)$ denotes the jump of $r^2$ between the two ends of
the handle.
In the multi--handle ensemble considered in later sections, the sum of
such contributions determines the effective WZW level that appears in
the boundary chiral action.
This underlies the anomaly--matching picture developed in
Sec.~\ref{sec:boundary}.

The detailed derivation of \eqref{eq:NY-total-derivative} from the
tetrad \eqref{eq:ref-tetrad-0}--\eqref{eq:ref-tetrad-3} and
\eqref{eq:phys-tetrad}--\eqref{eq:rot-matrix-factor} is straightforward
but lengthy; it is relegated to Appendix~B.

\subsection{Static versus spinning ans\"atze}
\label{subsec:ansatz-static-vs-spinning}

It is instructive to contrast the spinning ansatz above with more
naive alternatives.

If one sets $\omega=0$ and keeps only a static twist
$U(\psi)=\exp(i m\psi\,\sigma_3/2)$, the Nieh--Yan density
\eqref{eq:NY-total-derivative} loses the cross term proportional to
$\omega m$.
In such a purely twisted configuration the handle still carries a
flux $q$, but the Nieh--Yan density no longer exhibits the simple
total--derivative structure that makes the identification of an
effective WZW level transparent.

Conversely, if one tries to encode precession by letting the handle
axis itself rotate in time with respect to a fixed laboratory frame,
the tetrad becomes time--dependent already at the level of the axis
vector $e^1$.
This leads to an unnecessarily complicated TEGR energy functional and
obscures the separation between
\begin{itemize}
  \item the \emph{background} spinning handle
        (characterised by $q,m,\omega,r(\psi)$), and
  \item the \emph{small precession mode} $\varepsilon(t)$ that we
        introduce in Phase~2.
\end{itemize}

The present Spinning $SU(2)$ Handle Ansatz avoids both problems:
\begin{itemize}
  \item the axis $e^1$ is time--independent, so that a small tilt
        $\varepsilon(t)$ can be added cleanly on top of the spinning
        background, and
  \item the Nieh--Yan density acquires the exact--form structure
        \eqref{eq:NY-total-derivative}, which allows us to treat
        $\int_M\mathcal N$ as a bona fide contribution to an effective
        WZW level in the boundary theory.
\end{itemize}

For these reasons we adopt the spinning ansatz
\eqref{eq:ref-tetrad-0}--\eqref{eq:ref-tetrad-3} and
\eqref{eq:SU2-U}--\eqref{eq:rot-matrix-factor} as our
\emph{canonical microscopic geometry} for a single handle, and relegate
static variants and precessing laboratory--frame configurations to
Appendix~F.

