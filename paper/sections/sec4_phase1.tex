\section{Phase 1: Stability and energy scaling}
\label{sec:phase1}

In this section we place the Spinning $SU(2)$ handle ansatz of
Sec.~\ref{sec:ansatz} into the TEGR action of Sec.~\ref{sec:TEGR}, and
reduce the dynamics to the single radial degree of freedom $r(\psi)$ in a
one-dimensional mini-superspace.  We then show that pure TEGR already
admits a classically stable equal-radius handle solution with a radius
\begin{equation}
  r_0 \;\propto\; |q|^{1/2},
  \qquad
  \text{(for fixed non-zero twist number $m$)},
  \label{eq:r0-scaling-intro}
\end{equation}
and that the corresponding TEGR energy scales linearly with the absolute
value of the monopole charge $q$,
\begin{equation}
  E_{\text{TEGR}}(q)
  \;\propto\; |q|,
  \qquad
  \text{(again with fixed non-zero $m$)}.
\end{equation}
These results provide the $P$--even geometric background for the
parity-violating effects discussed in Phase~2 and the binding analysis
in Phase~3.

%------------------------------------------------------------
\subsection{Effective mini-superspace energy functional}
\label{subsec:phase1-effective}

We work in pure TEGR,
\begin{equation}
  S_{\text{TEGR}}
  \;=\;
  \frac{1}{2\kappa^2}
  \int d^4x\, e\, \mathbb{T},
\end{equation}
with Weitzenb\"ock connection and torsion scalar $\mathbb{T}$ as defined in
Sec.~\ref{sec:TEGR}.  On the local product manifold
$M \simeq \mathbb{R}_t \times S^1_\psi \times S^2$, the Spinning
$SU(2)$ handle ansatz of Sec.~\ref{sec:ansatz} reduces the geometric
degrees of freedom to a single radial function $r(\psi)$ measuring
the isotropic radius of the $S^2$ cross-section at each point
along the handle axis $S^1_\psi$.

After inserting the ansatz into $S_{\text{TEGR}}$, averaging over time,
and integrating over the $S^2$ cross-section, one obtains an effective
one-dimensional energy functional for $r(\psi)$,
\begin{equation}
  E[r]
  \;=\;
  \int_0^{L_\psi} d\psi\,
  \Big[
    A\,(r')^2 + V(r)
  \Big],
  \qquad
  r' := \frac{dr}{d\psi},
  \label{eq:effective-energy-functional}
\end{equation}
where $L_\psi$ denotes the coordinate length of the handle along the
$\psi$--direction and $A>0$ is a positive coefficient determined by the
TEGR normalisation and the detailed tetrad choice.
The coefficient $A$ arises from axial components of the torsion scalar
(schematically from terms of the form
$T^{\psi}{}_{r\psi} T_\psi{}^{r\psi}$, etc.) and is expected to be of
Planckian magnitude in a UV-complete theory.

The effective potential $V(r)$ for an equal-radius handle takes the form
\begin{equation}
  V(r)
  \;=\;
  \frac{B\,q^2}{r^2} + C\,m^2 r^2,
  \qquad B>0,\;C>0,
  \label{eq:effective-potential}
\end{equation}
where
\begin{itemize}
  \item $q \in \mathbb{Z}$ is the integer monopole charge associated with
        the torsion flux through the $S^2$ cross-section, as defined in
        Sec.~\ref{sec:ansatz};
  \item $m \in \mathbb{Z}$ is the twist number of the $SU(2)$ rotation
        along the handle axis;
  \item $B$ and $C$ are positive constants set by the TEGR coupling
        and the detailed geometry of the ansatz.
\end{itemize}
The first term, $\propto q^2/r^2$, originates from the flux contribution
to the torsion scalar and behaves as a repulsive ``core'' potential that
diverges for $r \to 0$.  The second term, $\propto m^2 r^2$, arises from
the uniform twist along $S^1_\psi$ and plays the rôle of a tension or
centrifugal contribution that diverges for $r \to \infty$.

Spin (self-rotation) $\omega$ does contribute to the torsion scalar, but
in the co-rotating frame implicit in the ansatz its $r$--dependent
contributions cancel between kinetic and centrifugal terms for static,
equal-radius configurations.  As a result, $\omega$ does not appear in
$V(r)$, and its rôle is deferred to the Nieh-Yan sector and the
precession dynamics discussed in later sections.

%------------------------------------------------------------
\subsection{Classical stability of the equal-radius handle}
\label{subsec:phase1-stability}

We now specialise to static equal-radius configurations
\begin{equation}
  r(\psi) = r_0 = \text{const.},
\end{equation}
for which the gradient term in
\eqref{eq:effective-energy-functional} vanishes and the energy reduces
to
\begin{equation}
  E[r_0]
  \;=\;
  \int_0^{L_\psi} d\psi\, V(r_0)
  \;=\;
  L_\psi\, V(r_0).
  \label{eq:E-equal-radius}
\end{equation}
The classical equilibrium radius is thus determined by the extremum
condition
\begin{equation}
  \frac{dV}{dr}\Big|_{r=r_0} = 0.
\end{equation}
Using \eqref{eq:effective-potential},
\begin{equation}
  \frac{dV}{dr}
  \;=\;
  -\,\frac{2B q^2}{r^3} + 2C m^2 r,
\end{equation}
so that the extremum satisfies
\begin{equation}
  C m^2 r_0^4 = B q^2
  \qquad\Longrightarrow\qquad
  r_0^4 = \frac{B q^2}{C m^2}.
  \label{eq:r0-fourth}
\end{equation}
For $q\neq 0$ and $m\neq 0$ this yields a finite, nonzero equilibrium
radius
\begin{equation}
  r_0
  \;=\;
  \left(\frac{B}{C}\right)^{1/4}
  \frac{|q|^{1/2}}{|m|^{1/2}},
  \label{eq:r0-explicit}
\end{equation}
up to a theory-dependent numerical factor $(B/C)^{1/4}$.  In particular,
for a fixed non-zero twist number $m$ the radius scales as
\begin{equation}
  r_0 \;\propto\; |q|^{1/2},
  \qquad
  (m \text{ fixed}),
\end{equation}
in agreement with the scaling quoted in
Eq.~\eqref{eq:r0-scaling-intro}.

To check that this extremum is a minimum, we compute the second
derivative of $V(r)$,
\begin{equation}
  \frac{d^2V}{dr^2}
  \;=\;
  \frac{6B q^2}{r^4} + 2C m^2.
\end{equation}
Evaluating at $r=r_0$ and using \eqref{eq:r0-fourth},
\begin{equation}
  \frac{d^2V}{dr^2}\Big|_{r=r_0}
  \;=\;
  \frac{6B q^2}{r_0^4} + 2C m^2
  \;=\;
  6C m^2 + 2C m^2
  \;=\;
  8C m^2 \;>\; 0,
\end{equation}
so the equal-radius configuration is indeed a local minimum of the
effective potential for any $q\neq 0$ and $m\neq 0$.

Small fluctuations $\delta r(\psi)$ around $r_0$ can be analysed by
expanding \eqref{eq:effective-energy-functional} to quadratic order,
\begin{equation}
  r(\psi) = r_0 + \delta r(\psi),
\end{equation}
which yields a Sturm-Liouville problem with a positive mass term
proportional to $d^2V/dr^2|_{r_0}$ and a gradient term controlled by
the coefficient $A$.  In the regime of interest for this geometric mechanism, $A$ is expected
to be of Planckian order in natural units, so that non-uniform Fourier
modes of $\delta r$ are heavy and strongly suppressed.  Consequently, in
the semiclassical regime relevant to our handle model the equal-radius configuration
$r(\psi)=r_0$ provides an excellent approximation to the true vacuum,
with radial oscillations confined to sub-Planckian amplitudes.
(A more detailed estimate of $A$ and the fluctuation spectrum is
deferred to Appendix~\ref{app:phase1}.)

%------------------------------------------------------------
\subsection{On-shell TEGR energy and linear scaling in $|q|$}
\label{subsec:phase1-energy-scaling}

We now evaluate the TEGR contribution to the energy on the equal-radius
solution $r(\psi)=r_0$ determined above.  Substituting
\eqref{eq:r0-fourth} into \eqref{eq:effective-potential}, we obtain
\begin{align}
  V(r_0)
  &= \frac{B q^2}{r_0^2} + C m^2 r_0^2
  \nonumber\\[2pt]
  &= \frac{B q^2}{\bigl(\tfrac{B}{C}\bigr)^{1/2} |q/m|}
     + C m^2 \bigl(\tfrac{B}{C}\bigr)^{1/2} \Bigl|\frac{q}{m}\Bigr|
  \nonumber\\[2pt]
  &= \sqrt{BC}\,|m q| + \sqrt{BC}\,|m q|
  \nonumber\\[2pt]
  &= 2\sqrt{BC}\,|m q|.
  \label{eq:V-r0}
\end{align}
The corresponding TEGR energy is
\begin{equation}
  E_{\text{TEGR}}(q,m)
  \;=\;
  L_\psi\, V(r_0)
  \;=\;
  2L_\psi \sqrt{BC}\,|m q|.
  \label{eq:TEGR-energy-on-shell}
\end{equation}
For fixed twist number $m\neq 0$ and handle length $L_\psi$ this implies
the linear scaling
\begin{equation}
  E_{\text{TEGR}}(q)
  \;\equiv\; E_{\text{TEGR}}(q,m_{\rm fixed})
  \;\approx\;
  \alpha\,|q|,
  \qquad
  \alpha := 2L_\psi \sqrt{BC}\,|m| > 0.
  \label{eq:E-TEGR-linear}
\end{equation}
Thus, within the Spinning $SU(2)$ handle ansatz, pure TEGR behaves as a
\emph{geometric mass term} that is linear in the absolute value of the
monopole charge $q$.  The twist number $m\neq 0$ is required for the
existence of a stable minimum and sets the overall energy scale, but it
does not modify the linear dependence on $|q|$.

In particular, the TEGR energy is insensitive to the sign of $q$; it is
even under parity and charge conjugation, and does not distinguish
between $q$ and $-q$.  The sign-sensitivity and chirality selection that
are essential for this framework therefore cannot come from the TEGR
sector alone.  They enter through the Nieh--Yan term and its effective
boundary coupling to the precession mode, which we analyse in Phase~2.
