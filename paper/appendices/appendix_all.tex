\appendix

%%%%%%%%%%%%%%%%%%%%%%%%%%%%%%%%%%%%%%%%%%%%%%%%%%%%%%%%%%%%%%%%%%%%%%%%%%%%%%
\section{Notation, Conventions, and Dimensional Analysis}
\label{app:notation}
%%%%%%%%%%%%%%%%%%%%%%%%%%%%%%%%%%%%%%%%%%%%%%%%%%%%%%%%%%%%%%%%%%%%%%%%%%%%%%

We work in natural Planck units
\begin{equation}
\hbar = c = 8\pi G = 1,
\end{equation}
so that length, time and inverse mass all have the same dimension,
\([L]=[T]=[M]^{-1}\).  The Planck length is denoted by
$\ell_{\rm Pl}$.

Throughout the paper we use the mostly–plus signature
$(-,+,+,+)$ and Latin indices $a,b,\dots$ for tangent–space indices,
Greek indices $\mu,\nu,\dots$ for spacetime indices.
The metric is obtained from the tetrad by
\begin{equation}
g_{\mu\nu} = e^a{}_\mu e^b{}_\nu \eta_{ab},
\qquad
\eta_{ab} = \mathrm{diag}(-1,1,1,1).
\end{equation}

\subsection{Physical parameters and their dimensions}

The basic parameters appearing in our handle ansatz are summarised
in Table~\ref{tab:dimensions}.  When needed we explicitly restore
powers of $\ell_{\rm Pl}$; otherwise we simply treat all dimensionful
quantities as measured in Planck units.

\begin{table}[h]
  \centering
  \begin{tabular}{lll}
    \hline
    Symbol & Meaning & Dimension \\
    \hline
    $q$         & torsional monopole charge (Nieh--Yan charge) & dimensionless ($\mathbb{Z}$) \\
    $m$         & twist winding number                          & dimensionless ($\mathbb{Z}$) \\
    $\omega$    & background spin frequency                     & $[T]^{-1}\sim [\ell_{\rm Pl}]^{-1}$ \\
    $r_0$       & handle radius                                 & $[L]\sim[\ell_{\rm Pl}]$ \\
    $L_\psi$    & length of compact direction $\psi$            & $[L]$ \\
    $\varepsilon$ & small precession angle                      & dimensionless \\
    $\theta_{\rm NY}$ & Nieh--Yan coupling                      & $[\ell_{\rm Pl}]^{-2}$ \\
    \hline
  \end{tabular}
  \caption{Physical parameters and their dimensions.}
  \label{tab:dimensions}
\end{table}

The TEGR action is
\begin{equation}
S_{\rm TEGR}
 = \frac{1}{2\kappa}\int d^4x\, e\,\mathbb{T},
\qquad
\kappa = 8\pi G = 1,
\end{equation}
where $e=\det(e^a{}_\mu)$ and $\mathbb{T}$ is the torsion scalar.
Since $S_{\rm TEGR}$ is dimensionless, we have
\([e\,d^4x]\sim[\ell_{\rm Pl}]^4\) and $[\mathbb{T}]\sim[\ell_{\rm Pl}]^{-2}$.
For a configuration of characteristic size $r_0$ this gives the
parametric estimate
\begin{equation}
E_{\rm TEGR}
\sim \ell_{\rm Pl}^{-1}\times (\text{dimensionless function of }q,m,\omega),
\end{equation}
consistent with the energy per handle scaling as $E\propto |q|$ in the
main text.

For the precession mode $\varepsilon(t)$ we eventually obtain an effective
action of the form
\begin{equation}
S_{\rm eff}^{(\varepsilon)}
= \frac{1}{2}\int dt\, k(q)\,\varepsilon(t)^2 + \cdots ,
\end{equation}
so $k(q)$ has dimension $[\ell_{\rm Pl}]^{-1}$.
In our ansatz this stiffness scales as
\begin{equation}
k(q)
\sim m^2\omega^2 r_0^2 L_\psi
\sim \omega^2 |q|,
\end{equation}
where we used $r_0^2\propto |q|$ (with $m$ fixed).

The Nieh--Yan term is
\begin{equation}
S_{\rm NY}
  = \theta_{\rm NY}\int_M \mathcal{N},
\end{equation}
with $\mathcal{N}$ the Nieh--Yan density.  For a single handle the
integral scales as $\int_M\mathcal{N}\sim q\,\Delta r^2\sim
q\,\ell_{\rm Pl}^2$, so that $[\theta_{\rm NY}]=[\ell_{\rm Pl}]^{-2}$,
in agreement with Table~\ref{tab:dimensions}.

All coefficients $A,B,C,\dots$ that appear in the 1D effective
description in Phase~1 are understood as $\mathcal{O}(1)$ numerical
constants in these Planck units.



%%%%%%%%%%%%%%%%%%%%%%%%%%%%%%%%%%%%%%%%%%%%%%%%%%%%%%%%%%%%%%%%%%%%%%%%%%%%%%
\section{Explicit Tetrad: Background Spin/Twist and Precession}
\label{app:tetrad}
%%%%%%%%%%%%%%%%%%%%%%%%%%%%%%%%%%%%%%%%%%%%%%%%%%%%%%%%%%%%%%%%%%%%%%%%%%%%%%

\subsection{Static monopole reference tetrad}

On $\mathbb{R}_t\times S^1_\psi\times S^2$ we use coordinates
$(t,\psi,\theta,\phi)$.
A convenient static reference tetrad carrying monopole charge $q$
is
\begin{align}
\tilde e^0 &= dt, \label{eq:ref-static-0}\\
\tilde e^1 &= d\psi, \label{eq:ref-static-1}\\
\tilde e^2 &= r_0\, d\theta, \label{eq:ref-static-2}\\
\tilde e^3 &= r_0\sin\theta\,\bigl[d\phi + q(1-\cos\theta)\,d\psi\bigr].
\label{eq:ref-static-3}
\end{align}
This tetrad reproduces the usual monopole gauge potential on $S^2$
in the $(\theta,\phi)$ directions and an $S^1$ fibre parametrised by
$\psi$.

\subsection{Background spinning--twisted tetrad}

The physical background used in the main text combines
\emph{spin} around the handle axis and a \emph{twist} along
$S^1_\psi$.
At the level of the internal $SU(2)$ frame this can be represented
schematically by
\begin{equation}
U(t,\psi)=
\exp\!\left( \frac{i}{2}\omega t\,\sigma_3\right)
\exp\!\left( \frac{i}{2}m\psi\,\sigma_3\right),
\end{equation}
with $\sigma_3$ a Pauli matrix.
The background tetrad $e^a{}_\mu{}_{\rm bg}$ is obtained by acting
with the corresponding $SO(3)$ rotation on the spatial triad
$\tilde e^{i}$ ($i=1,2,3$).  The explicit expressions are not needed
here; the important point is that they preserve axial symmetry and
encode the monopole charge $q$, twist $m$ and spin frequency $\omega$.

\subsection{Precession as a small axis tilt}

To model precession we introduce a slow, spatially uniform tilt
of the handle axis.  In a frame adapted to the background, the
tilt can be implemented by an infinitesimal $SO(3)$ rotation that
mixes the $e^1$ and $e^3$ directions:
\begin{equation}
e^a = \Lambda^a{}_b(\varepsilon(t))\,e^b_{\rm bg},
\qquad
\Lambda(\varepsilon) =
\begin{pmatrix}
1 & 0 & 0 & 0 \\
0 & \cos\varepsilon & 0 & \sin\varepsilon \\
0 & 0 & 1 & 0 \\
0 & -\sin\varepsilon & 0 & \cos\varepsilon
\end{pmatrix},
\end{equation}
with $\varepsilon(t)\ll 1$.
To first order in $\varepsilon$ one has
\begin{align}
e^1 &\simeq e^1_{\rm bg} + \varepsilon(t)\, e^3_{\rm bg},\\
e^3 &\simeq e^3_{\rm bg} - \varepsilon(t)\, e^1_{\rm bg},
\end{align}
while $e^0$ and $e^2$ are unchanged.
Thus the handle axis, originally aligned with $d\psi$, acquires a
small time–dependent tilt of order $\varepsilon(t)$.

In the explicit stiffness calculation of
Appendix~\ref{app:phase3-calc} it is technically convenient to
work in a rotated spatial frame in which the same physical precession
is represented as a rotation in the $(e^2,e^3)$–plane while $e^1$
is kept fixed.  This change of frame does not affect any physical
observable, but simplifies the algebra in the SymPy implementation.



%%%%%%%%%%%%%%%%%%%%%%%%%%%%%%%%%%%%%%%%%%%%%%%%%%%%%%%%%%%%%%%%%%%%%%%%%%%%%%
\section{Phase 1: Detailed Derivation of the Effective Radial Potential}
\label{app:phase1}
%%%%%%%%%%%%%%%%%%%%%%%%%%%%%%%%%%%%%%%%%%%%%%%%%%%%%%%%%%%%%%%%%%%%%%%%%%%%%%

\subsection{TEGR action on the spinning--twisted ansatz}

Evaluating the TEGR action on the spinning--twisted background
tetrad of Sec.~\ref{app:tetrad} and imposing the equal–radius ansatz
$r(\psi)=r_0$ yields an effective one–dimensional functional of the
form
\begin{equation}
S_{\rm TEGR}
= \int dt\int_0^{L_\psi} d\psi \,
\left[
 A\,(\partial_\psi r)^2 + V(r)
\right],
\end{equation}
where $A>0$ is a numerical coefficient and $V(r)$ is an effective
radial potential.  For the equal–radius sector we may set
$\partial_\psi r=0$, so that the energy per unit $\psi$–length
reduces to
\begin{equation}
\mathcal{E} = \int_0^{L_\psi} d\psi\, V(r_0)
 = L_\psi\,V(r_0).
\end{equation}

\subsection{Dominant contributions to $V(r)$}

A straightforward but algebraically lengthy evaluation of the torsion
scalar on this ansatz shows that the dominant contributions to
$V(r)$ are
\begin{align}
V_{\rm core}(r) &= B\,\frac{q^2}{r^2},\\
V_{\rm twist}(r) &= C\,m^2 r^2,
\end{align}
with $B,C>0$ numerical coefficients of order unity in the Planck
units defined in Appendix~\ref{app:notation}.  The first term is
supported by the monopole–like torsion generated by $q$, while the
second term comes from the anisotropic ``twist'' gradients set by
$m$ along the fibre direction.

Higher–derivative corrections and subleading powers of $r$ are
suppressed for the nearly uniform configurations of interest and are
therefore dropped at this stage.

\subsection{Minimisation and stable radius}

The total potential is
\begin{equation}
V(r) = V_{\rm core}(r) + V_{\rm twist}(r)
     = B\,\frac{q^2}{r^2} + C\,m^2 r^2.
\end{equation}
Minimising with respect to $r$ gives
\begin{equation}
\frac{dV}{dr} = -2B\,\frac{q^2}{r^3} + 2C m^2 r = 0,
\end{equation}
so that the equilibrium radius satisfies
\begin{equation}
r_0^4 = \frac{B}{C}\,\frac{q^2}{m^2}
\qquad\Rightarrow\qquad
r_0^2 \propto \frac{|q|}{|m|}.
\end{equation}
The second derivative at the extremum,
\begin{equation}
\frac{d^2V}{dr^2}\bigg|_{r=r_0}
 = 6B\,\frac{q^2}{r_0^4} + 2C m^2 > 0,
\end{equation}
is strictly positive, confirming that this configuration is a
classical minimum.

Substituting $r_0$ back into $V(r_0)$ we obtain
\begin{equation}
E_{\rm TEGR}(q)
 = L_\psi V(r_0)
 = \alpha\,|q|,
\qquad
\alpha = 2\sqrt{BC}\,|m|\,L_\psi > 0,
\end{equation}
showing that the TEGR energy is proportional to $|q|$, as stated in
Sec.~4 of the main text.  The equal–radius approximation is
self–consistent as long as the curvature and torsion scales are
super–Planckian, which is the regime of interest in this toy model.



%%%%%%%%%%%%%%%%%%%%%%%%%%%%%%%%%%%%%%%%%%%%%%%%%%%%%%%%%%%%%%%%%%%%%%%%%%%%%%
\section{Nieh--Yan Boundary Term and Precession Coupling}
\label{app:phase2}
%%%%%%%%%%%%%%%%%%%%%%%%%%%%%%%%%%%%%%%%%%%%%%%%%%%%%%%%%%%%%%%%%%%%%%%%%%%%%%

Starting from the explicit spinning--twisted tetrad, one finds that
the Nieh--Yan density on a single handle takes the exact form
\begin{equation}
\mathcal{N}
 = q\,\omega m\,
   d\!\bigl(r(\psi)^2\bigr)\wedge dt\wedge
   \sin\theta\,d\theta\wedge d\phi
 \quad
 (\text{up to a positive numerical factor}),
\end{equation}
where $r(\psi)$ is the local radius of the $S^2$ cross–section and
$\omega$ is the background spin frequency.  Integrating over the
two–sphere and along the handle we obtain
\begin{equation}
\int_M \mathcal{N}
 = 4\pi\,q\,\omega m\,\Delta r^2,
\qquad
\Delta r^2 \equiv
 r^2(\psi_{\rm end}) - r^2(\psi_{\rm start}),
\end{equation}
again up to an overall positive factor that we absorb into the
definition of the coupling $\theta_{\rm NY}$.

For an exactly equal–radius configuration we have $\Delta r^2=0$
and therefore
\begin{equation}
\int_M\mathcal{N}=0
\quad\Rightarrow\quad
S_{\rm NY}=0.
\end{equation}
This confirms the statement in the main text that the spinning
background by itself remains parity even; parity violation arises
only once we take into account boundary effects associated with
precession.

\subsection{Precession--induced boundary mismatch}

When the handle precesses with a small, slowly varying angle
$\varepsilon(t)$, the junction between the microscopic handle and
the external ``bulk'' geometry develops a small mismatch.
To first order in $\varepsilon$ the difference in the effective
$r^2$ between the two ends of the handle can be parameterised as
\begin{equation}
\Delta r^2
\;\longrightarrow\;
\Delta r^2 + \beta\,\varepsilon(t),
\end{equation}
where $\beta>0$ is a model–dependent constant encoding how the
junction geometry responds to a uniform tilt.\footnote{A concrete
realisation of this behaviour can be given in a multi--handle
geometry where the external metric is kept fixed while the
microscopic handle is rotated.  The precise value of $\beta$ is not
needed in the main text; only its sign matters.}
Substituting into the Nieh--Yan term we obtain a linear contribution
to the effective action for $\varepsilon(t)$,
\begin{equation}
S_{\rm NY}^{(1)}
 = \theta_{\rm NY}\int dt\,
 \Lambda_q\,\varepsilon(t),
\qquad
\Lambda_q
 = {\sf c}_{\rm NY}\,q,
\end{equation}
with
${\sf c}_{\rm NY} \sim \theta_{\rm NY}\,\omega m\beta$.
We choose the sign convention such that
\begin{equation}
\Lambda_q = -\gamma\,q,
\qquad \gamma>0,
\end{equation}
so that the effective potential
$V_{\rm eff}(\varepsilon;q)$ is minimised at a small tilt
$\varepsilon_*(q)$ whose sign is locked to the sign of $q$,
as discussed in Sec.~5.

\begin{figure}[t]
  \centering
  \includegraphics[width=0.9\linewidth]{figs/Fig1_Junction_Mismatch.png}
  \caption{Schematic representation of the Nieh--Yan boundary
  mechanism.  Panel (a): aligned handle ($\varepsilon=0$) with
  identical junctions at both ends, $\Delta r^2=0$ and no Nieh--Yan
  contribution.  Panel (b): precessing handle ($\varepsilon\neq 0$);
  the junction at the upper end no longer matches the external
  geometry (red contour), producing a small shift
  $\Delta r^2\to\Delta r^2+\beta\varepsilon$ and hence a linear term
  $\Lambda_q\varepsilon$ in the effective action via the Nieh--Yan
  density.}
  \label{fig:junction}
\end{figure}

The key point is that the entire parity–violating effect is of
\emph{purely boundary origin}.  No bulk contribution to the Nieh--Yan
density is required once the equal–radius condition is imposed.



%%%%%%%%%%%%%%%%%%%%%%%%%%%%%%%%%%%%%%%%%%%%%%%%%%%%%%%%%%%%%%%%%%%%%%%%%%%%%%
\section{Complete Calculation of the Precession Stiffness $k(q)$}
\label{app:phase3-calc}
%%%%%%%%%%%%%%%%%%%%%%%%%%%%%%%%%%%%%%%%%%%%%%%%%%%%%%%%%%%%%%%%%%%%%%%%%%%%%%

\subsection{Tetrad with precession in a convenient frame}

For the explicit computation of the stiffness $k(q)$ we adopt a
frame in which the precession is realised as a rotation in the
$(e^2,e^3)$–plane, while $e^1$ is kept fixed.  This frame is related
by a rigid spatial rotation to the tilted--axis picture of
Appendix~\ref{app:tetrad} and is physically equivalent.

Let $\tilde e^a$ denote the aligned tetrad of
Eqs.~\eqref{eq:ref-tetrad-0}--\eqref{eq:ref-tetrad-3} in the main
text.  The precessing tetrad is
\begin{equation}
e^a = \Lambda^a{}_b(\varepsilon(t))\,\tilde e^b,
\qquad
\Lambda(\varepsilon)=
\begin{pmatrix}
1 & 0 & 0 & 0 \\
0 & 1 & 0 & 0 \\
0 & 0 & \cos\varepsilon & -\sin\varepsilon \\
0 & 0 & \sin\varepsilon & \cos\varepsilon
\end{pmatrix}.
\end{equation}
Expanding for small $\varepsilon$,
\begin{equation}
\cos\varepsilon\simeq 1-\frac{\varepsilon^2}{2},
\qquad
\sin\varepsilon\simeq \varepsilon,
\end{equation}
gives
\begin{align}
e^1 &= \tilde e^1,\\
e^2 &\simeq \tilde e^2 + \varepsilon\,\tilde e^3
        -\frac{\varepsilon^2}{2}\,\tilde e^2,\\
e^3 &\simeq \tilde e^3 - \varepsilon\,\tilde e^2
        -\frac{\varepsilon^2}{2}\,\tilde e^3.
\end{align}

\subsection{Torsion tensor to $\mathcal{O}(\varepsilon^2)$}

Using the Weitzenböck connection
$\Gamma^\lambda{}_{\mu\nu} = e_a{}^\lambda \partial_\nu e^a{}_\mu$,
the torsion tensor is
\begin{equation}
T^\lambda{}_{\mu\nu}
 = \Gamma^\lambda{}_{\nu\mu} - \Gamma^\lambda{}_{\mu\nu}.
\end{equation}
In an orthonormal frame this can be written as
$T^a = de^a + \omega^a{}_b\wedge e^b$; in the teleparallel gauge
$\omega^a{}_b=0$ so $T^a=de^a$.

Carrying out the expansion to second order in $\varepsilon$ and
keeping only the terms relevant for the precession mode, one finds
non–vanishing corrections such as
\begin{align}
\delta T^1{}_{\theta\phi} &= \varepsilon^2\,
  2q r_0 \sin\theta,\\
\delta T^2{}_{\psi\theta} &= \varepsilon^2\,
  m\omega r_0 \sin\theta,\\
\delta T^3{}_{\psi\theta} &= -\varepsilon^2\,
  m\omega r_0 \cos\theta,
\end{align}
together with similar terms related by symmetry.  All other components
either vanish or contribute only at higher orders in $\varepsilon$.

\subsection{Quadratic torsion scalar and symmetry protection}

The torsion scalar is
\begin{equation}
\mathbb{T}
 = \frac{1}{4}T^{\rho\mu\nu}T_{\rho\mu\nu}
 + \frac{1}{2}T^{\rho\mu\nu}T_{\nu\mu\rho}
 - T_\rho T^\rho,
\qquad
T_\rho = T^\mu{}_{\rho\mu}.
\end{equation}
Inserting the expanded tetrad and keeping only the
$\mathcal{O}(\varepsilon^2)$ terms, one can schematically write
\begin{equation}
\mathbb{T}^{(2)}
 = \varepsilon^2\Bigl[m^2\omega^2 r_0^2
   + (\text{terms}\propto q^2,\; q m\omega,\;\dots)\Bigr].
\end{equation}

A crucial point, confirmed both analytically and in the SymPy
implementation, is that \emph{all terms proportional to
$q^2\varepsilon^2$ cancel identically}.  This cancellation is
protected by the spherical symmetry of the monopole background:
a rigid rotation of a spherically symmetric configuration cannot
generate a restoring force proportional to $q^2$.  What survives at
$\mathcal{O}(\varepsilon^2)$ is the interference between the
anisotropic twist ($m$) and spin ($\omega$) sectors.

After the cancellation we are left with
\begin{equation}
\mathbb{T}^{(2)} = \varepsilon^2\,m^2\omega^2 r_0^2.
\end{equation}

\subsection{Integration and effective stiffness}

Integrating over the two–sphere and along the handle we obtain
\begin{align}
\int d^4x\, e\,\mathbb{T}^{(2)}
 &= \varepsilon^2 m^2\omega^2 r_0^2 L_\psi
    \int_0^\pi\sin\theta\,d\theta
    \int_0^{2\pi} d\phi \nonumber\\
 &= \varepsilon^2 m^2\omega^2 r_0^2 L_\psi
    \times 4\pi.
\end{align}
Up to an overall numerical factor that depends on conventions and
normalisation of the tetrad, this contribution induces an effective
action
\begin{equation}
S_{\rm TEGR}^{(2)}
 = \frac{1}{2}\int dt\,k(q)\,\varepsilon(t)^2,
\qquad
k(q) = \kappa_k\,L_\psi m^2\omega^2 r_0^2,
\end{equation}
with $\kappa_k=\mathcal{O}(1)$.  Using $r_0^2\propto |q|$ from
Appendix~\ref{app:phase1} we obtain the scaling
\begin{equation}
k(q)\propto \omega^2 |q|
\qquad\Rightarrow\qquad
\gamma = 1
\end{equation}
in the notation of the main text.

In the compact SymPy implementation described in the supplementary
material the same calculation yields a specific numerical value
$\kappa_k = 8\pi/9$ in our units.  This difference with respect to
the analytic coefficient $4\pi$ obtained above originates from a
slightly different normalisation of the frame fields and does not
affect the $|q|$– and $\omega$–scaling.  Since the main text never
uses the absolute value of $\kappa_k$, we keep it symbolic here.

\begin{lstlisting}[caption={SymPy code fragment used for the
  precession stiffness calculation.  The full listing is provided in
  the supplementary material.},label={code:appe}]
# Python/SymPy code fragment – full listing in supplementary material
e2 = cos(eps)*e2_tilde + sin(eps)*e3_tilde
e3 = -sin(eps)*e2_tilde + cos(eps)*e3_tilde
T2 = de2  # teleparallel gauge: T^a = de^a
T3 = de3
T2_quad = T2.subs(eps, 0).series(eps, 0, 3).coeff(eps**2)
# → shows q**2 terms cancel; only m*omega terms survive
\end{lstlisting}



%%%%%%%%%%%%%%%%%%%%%%%%%%%%%%%%%%%%%%%%%%%%%%%%%%%%%%%%%%%%%%%%%%%%%%%%%%%%%%
\section{Universality of the $\gamma=1$ Scaling}
\label{app:universality}
%%%%%%%%%%%%%%%%%%%%%%%%%%%%%%%%%%%%%%%%%%%%%%%%%%%%%%%%%%%%%%%%%%%%%%%%%%%%%%

The result $k(q)\propto \omega^2|q|$ obtained above might at first
sight appear to be an artefact of the specific $SU(2)$ parametrisation
used for the microscopic handle.  In this appendix we argue that,
within the class of axially symmetric spinning/twisted handles
considered in this paper, the scaling with $\gamma=1$ is in fact
generic.

The static energy is controlled by two competing contributions:
\begin{itemize}
\item a ``core'' term associated with the torsional monopole flux,
      scaling as $V_{\rm core}\sim q^2/r_0^2$;
\item a ``twist'' term associated with gradients along the fibre,
      scaling as $V_{\rm twist}\sim m^2 r_0^2$.
\end{itemize}
These depend only on the topological charges $(q,m)$ and on the
radius $r_0$, but not on the detailed choice of tetrad within a
given symmetry class.  Minimising
$V_{\rm core}+V_{\rm twist}$ therefore always gives
$r_0^2\propto |q|/|m|$ up to an $\mathcal{O}(1)$ factor, independently
of the microscopic parametrisation.

The precession stiffness $k(q)$ originates from the \emph{dynamical}
response of the twisted/spinning configuration to a small tilt.
In any axially symmetric handle with monopole flux $q$, twist $m$
and spin $\omega$, the torsion scalar contains terms of the schematic
form
\begin{equation}
\mathbb{T} \supset \omega m\,(\text{spatial rotation terms}),
\end{equation}
whose quadratic response to a uniform tilt produces an energy density
proportional to $m^2\omega^2$.
Integrating this density over the cross–section of area
$\sim r_0^2$ and along the handle then yields
\begin{equation}
k(q)\propto \omega^2 m^2 r_0^2
          \propto \omega^2 |q|.
\end{equation}
Thus, within this axially symmetric class, the exponent
$\gamma=1$ is a robust consequence of (i) the $q^2/r_0^2$ vs.
$m^2 r_0^2$ competition that fixes $r_0^2\propto |q|$, and
(ii) the fact that the precession mode couples to the twist and spin,
not directly to the monopole flux.

We do \emph{not} claim that $\gamma=1$ holds as a rigorous theorem
for arbitrary teleparallel configurations.  Extending the analysis
to more general, non–axially symmetric handles is left as an open
problem for future work.



%%%%%%%%%%%%%%%%%%%%%%%%%%%%%%%%%%%%%%%%%%%%%%%%%%%%%%%%%%%%%%%%%%%%%%%%%%%%%%
\section{The Static Case ($\omega=0$): Phenomenological Sterility}
\label{app:static}
%%%%%%%%%%%%%%%%%%%%%%%%%%%%%%%%%%%%%%%%%%%%%%%%%%%%%%%%%%%%%%%%%%%%%%%%%%%%%%

The spinning ansatz used in the main text is technically more
involved than the purely static handle, but it is precisely the
background spin $\omega\neq 0$ that makes the microscopic handle
phenomenologically interesting.  In this appendix we briefly
summarise what happens when the spin is switched off.

\subsection{Vanishing linear term in the effective potential}

The Nieh--Yan contribution to the effective action for a single
handle is schematically
\begin{equation}
S_{\rm NY}
 \propto \theta_{\rm NY}\,q\,\omega m\,\Delta r^2,
\end{equation}
with $\Delta r^2$ the difference in $r^2$ between the two ends of
the handle (see Appendix~\ref{app:phase2}).  When $\omega=0$ this
term vanishes exactly, regardless of the value of $\Delta r^2$.
The effective potential $V_{\rm eff}(\varepsilon;q)$ therefore has
no linear term in $\varepsilon$ and remains parity even, yielding
no geometric mechanism for chirality selection.

\subsection{Vanishing topological coupling to statistics}

More fundamentally, the Nieh--Yan density takes the form
\begin{equation}
\mathcal{N}
 \propto q\,\omega m\,
  d(r^2)\wedge dt\wedge\text{(angular 2--form)}.
\end{equation}
When $\omega=0$ the $dt$ component disappears and the spacetime
integral of $\mathcal{N}$ reduces to a trivial surface term that
does not generate a non–zero topological coupling.
In particular, the heuristic link between the Nieh--Yan invariant
and a Wess--Zumino--Witten term controlling particle statistics
(used as motivation in the main text) is absent in the static case,
so the handle does not provide a natural geometric origin for
half–integer spin states.

\subsection{Mechanical stability vs.\ phenomenological sterility}

The configuration with $\omega=0$ remains mechanically stable:
the static balance between $V_{\rm core}$ and $V_{\rm twist}$ is
unchanged, and the stiffness of small geometric deformations is
still set by $k(q)\propto m^2 r_0^2 L_\psi$ (which is non--zero
independently of $\omega$, see Appendix~\ref{app:phase3-calc}).
In this sense the static handle is not catastrophically unstable.

However, precisely the phenomena that motivated our construction in
the first place --- chirality selection, a possible link to
fermionic statistics, and near–critical binding of composite states
--- all rely on the interplay between spin $\omega$ and the Nieh--Yan
boundary term.  When $\omega=0$ this interplay is absent, and the
handle becomes phenomenologically inert.

The non–zero background spin $\omega$ should therefore be viewed not
as a mere technical decoration of the ansatz, but as the crucial
ingredient that ``breathes phenomenological life'' into the
microscopic handle.
