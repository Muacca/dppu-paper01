\documentclass[12pt]{article}

\usepackage[a4paper,margin=2.5cm]{geometry}
\usepackage{amsmath,amssymb,amsfonts}
\usepackage{bm}
\usepackage{graphicx}
\usepackage{listings}
\usepackage{hyperref}
\usepackage[numbers,sort&compress]{natbib}

\title{Microscopic Handles in Teleparallel Gravity:\\
Toward a Geometric Origin of Fermionic Statistics,\\
Chirality, and Critical Binding}

\author{Muacca\\
\texttt{muacca@dmwp.jp}}
\date{\today}

\begin{document}

\maketitle

\begin{abstract}
We propose a geometric mechanism in four-dimensional teleparallel gravity (TEGR)
that offers a unified perspective on fermionic statistics, chirality, and the
near-critical binding of composite states. We consider spacetime containing
microscopic handles of topology $S^2\times S^1$, carrying an integer torsional
monopole charge $q$. Using a ``Spinning SU(2)'' tetrad ansatz, we derive three
interconnected results.

(i) \textbf{Stability:} The pure TEGR sector admits a classically stable handle
radius with an on-shell energy $E \simeq \alpha |q|$, providing a parity-even
geometric mass term. (ii) \textbf{Chirality:} The Nieh--Yan boundary term
produces a parity-odd contribution to the effective action. A slow precession
mode develops a chiral tilt whose direction is locked to the sign of $q$,
acting as a geometric mechanism for chirality selection. This construction is
motivated by the known connection between the Nieh--Yan invariant and
Wess--Zumino--Witten terms that govern particle statistics in Skyrmion-like
models. (iii) \textbf{Classical criticality:} The stiffness of the precession
mode scales as $k(q) \propto \omega^2 |q|$, placing the system at a
\emph{classical} critical point where the binding energy per unit charge is
independent of $q$ to leading order.

Consequently, subleading corrections could in principle render composite
states with $|q| \ge 2$ (in particular $|q|=3$) energetically competitive with
separated unit-charge handles. We regard the present construction as a
teleparallel toy model that establishes the classical critical point
$\gamma=1$ from geometry alone; the impact of quantum and geometric
corrections on an effective exponent $\gamma_{\mathrm{eff}}$ will be analysed
in a separate work.
\end{abstract}

% \tableofcontents  % 必要なら有効化
% \newpage

%-----------------------------
% Main text
%-----------------------------

\section*{Microscopic Handles in Teleparallel Gravity:\\
Toward a Geometric Origin of Fermionic Statistics, Chirality, and Critical Binding}

% TODO: Abstract will be written after the main body is more settled.

\section{Introduction}
\label{sec:intro}

The Standard Model of particle physics provides an extremely successful
description of microscopic phenomena, yet several of its key structural
features remain conceptually opaque.  Among these are
(i) the spin--statistics connection, in particular the fermionic
statistics of quarks and leptons,
(ii) the chiral, $V\!-\!A$ structure of weak interactions, and
(iii) the special role of color triplets and color-singlet
three-body bound states (baryons).
In the conventional framework these properties are encoded
in the local gauge and matter content of the theory, but they are
not obviously explained by the large-scale geometry of spacetime itself.

Most attempts to derive such properties from geometry have proceeded
by enlarging spacetime, e.g.\ via higher-dimensional Kaluza--Klein
constructions or string theory.  In this work we explore a different,
minimalistic route: we remain in four dimensions, but we allow spacetime
to carry a non-trivial microscopic topology and non-zero torsion.
Specifically, we work within the teleparallel equivalent of general
relativity (TEGR), where gravity is encoded in torsion rather than curvature,
and we assume that the spatial manifold contains a large number of
microscopic handles with topology $S^2 \times S^1$.
The central question we ask is whether such a microscopic handle
structure can provide a geometric origin for the above features of
particle physics.

\subsection{Motivation}
\label{subsec:intro-motivation}

The starting point of our proposal is the well-known fact that
topological solitons in non-linear sigma models can carry quantum
numbers such as baryon number, and that their statistics can be
affected by Wess--Zumino--Witten (WZW) terms in the effective action.
In particular, Witten showed that in an $SU(2)$ Skyrme model with
WZW level $k_{\mathrm{WZ}}$, a skyrmion behaves as a fermion when
$k_{\mathrm{WZ}}$ is odd and as a boson when $k_{\mathrm{WZ}}$ is even
\cite{Witten:1983tw}.
At the same time, the Nieh--Yan four-form is known to contribute
to the chiral anomaly on spaces with torsion, and thus can in principle
induce an effective WZW-like topological term for an $SU(2)$-valued field
coupled to torsion \cite{Chandia:1997hu}.

These observations suggest a possible chain of relations:
\begin{equation}
  \text{Nieh--Yan density} \;\longrightarrow\;
  \text{effective WZW-like level} \;\longrightarrow\;
  \text{statistics and internal quantum numbers of solitons}.
\end{equation}
In the usual treatments, the $SU(2)$ field is a matter field defined on
an otherwise smooth background.  In contrast, we will consider the
possibility that the same $SU(2)$ structure is encoded in the local
frame of a microscopic handle of spacetime itself.
This leads naturally to the idea that each microscopic handle might behave
as a solitonic defect whose topological charge is measured by the
Nieh--Yan invariant, and whose statistics and chirality are governed
by an induced topological term in the effective action.

To make this idea precise, we restrict attention---in this first paper---
to the simplest possible setting:
\begin{itemize}
  \item the gravitational sector is pure TEGR, with no higher-curvature
        corrections;
  \item the material content is summarized by an effective torsion
        background admitting a Dirac-type coupling;
  \item we focus on a single microscopic handle, modeled locally as
        $M \simeq \mathbb{R}_t \times S^1_\psi \times S^2_{(\theta,\phi)}$,
        and study its energetics and symmetry-breaking patterns.
\end{itemize}
Our goal is not to reproduce the full Standard Model, but to show that
a single microscopic handle in teleparallel gravity can already exhibit
three key features reminiscent of particle physics:
(i) a stable localized configuration with energy linear in an integer
topological charge $q$,
(ii) spontaneous parity breaking associated with a precession mode,
and (iii) a near-critical scaling of the stiffness that favors
certain composite charge states.

% === Grok patch: make the role of spin parameter explicit
We note in passing that purely static (non-spinning) handle configurations,
with vanishing spin parameter $\omega = 0$ (to be introduced explicitly
in Sec.~\ref{sec:ansatz}), fail to produce a stable minimum of the
effective potential; the spinning degree of freedom is therefore
physically indispensable (see Appendix~\ref{app:static} for details).

\subsection{Teleparallel framework and microscopic handles}
\label{subsec:intro-framework}

Teleparallel gravity replaces the Levi-Civita connection of general
relativity by a curvature-free connection with non-vanishing torsion.
The fundamental variable is a tetrad field $e^a{}_\mu$ whose determinant
we denote by $e = \det(e^a{}_\mu)$.
The metric is recovered in the usual way,
$g_{\mu\nu} = \eta_{ab} e^a{}_\mu e^b{}_\nu$,
while the connection is chosen such that its curvature vanishes
identically and all gravitational information resides in the torsion.
The corresponding Lagrangian density is built from a scalar $\mathbb{T}$
quadratic in the torsion tensor and is dynamically equivalent to the
Einstein--Hilbert action up to a total divergence.

In the scenario considered here, the large-scale metric
can be close to that of a spatial three-torus,
but the spatial topology is refined by the presence of many microscopic
handles.  Schematically one may think of the spatial slice as
\begin{equation}
  \Sigma \;\simeq\;
  T^3 \;\#\;
  \Bigl(\#_k \bigl(S^2 \times S^1\bigr)\Bigr),
\end{equation}
where $\#$ denotes the connected sum and each factor $S^2 \times S^1$
represents a microscopic handle.
In this first paper we assume that the handles are well separated and
weakly interacting, so that it is meaningful to focus on a single
representative handle and treat the rest as spectators.

Concretely, we select a four-dimensional region
\begin{equation}
  M \;\simeq\; \mathbb{R}_t \times S^1_\psi \times S^2_{(\theta,\phi)},
\end{equation}
with coordinates $x^\mu = (t,\psi,\theta,\phi)$ adapted to the handle.
We will construct an explicit tetrad on $M$ that encodes both a twist
along the $S^1_\psi$ direction and a rigid spin about the handle axis.
The resulting torsion yields a Nieh--Yan density whose integral over $M$
defines an integer-valued topological charge $q \in \mathbb{Z}$.
The main task of the present work is to analyze the energetics and
small deformations of such a configuration and to identify how
$q$ controls its stability, parity properties, and possible composite
states.

\subsection{Main results}
\label{subsec:intro-main-results}

Before entering the detailed construction, it is useful to summarize
the main results obtained in this paper.

\paragraph{(1) Stability and energy scaling (Phase~1).}
Using a ``Spinning $SU(2)$ Handle Ansatz'' for the tetrad on
$M \simeq \mathbb{R}_t \times S^1 \times S^2$, we show that the
TEGR action admits a class of static configurations in which the
handle radius $r$ is constant along the $S^1$ direction.
The effective one-dimensional energy functional for the radial profile
has the schematic form
\begin{equation}
  E[r] \;=\; \int_0^{2\pi} d\psi\,
  \bigl[ A (r')^2 + B\, q^2 / r^2 + C\, m^2 r^2 \bigr],
\end{equation}
where $q \in \mathbb{Z}$ is the Nieh--Yan charge and $m$ is an integer
twist number.
Minimizing this functional yields a stable radius
\begin{equation}
  r_0 \;\propto\; \bigl| q/m \bigr|^{1/2},
\end{equation}
and the on-shell TEGR energy scales linearly with the absolute value
of the charge,
\begin{equation}
  E_{\mathrm{TEGR}}(q) \;\propto\; |q|.
\end{equation}
The detailed derivation is given in Sec.~\ref{sec:phase1} and
Appendix~C.

\paragraph{(2) Spontaneous parity breaking (Phase~2).}
We then introduce a small precession mode $\varepsilon(t)$ describing
a tilt of the handle axis.
Expanding the TEGR action to quadratic order in $\varepsilon$ yields
an effective ``stiffness'' term
$\frac{1}{2} k(q)\, \varepsilon^2$ with $k(q) > 0$.
In the presence of a Nieh--Yan term, however, the handle boundary
contributes a term linear in $\varepsilon$,
$V_{\mathrm{NY}} \sim - \Lambda_q\, \varepsilon$,
where $\Lambda_q$ is proportional to the Nieh--Yan charge $q$ and
to an effective coupling $\theta_{\mathrm{NY}}$.
The resulting effective potential
\begin{equation}
  V_{\mathrm{eff}}(\varepsilon)
  \;=\;
  \tfrac{1}{2} k(q)\, \varepsilon^2 - \Lambda_q\, \varepsilon
\end{equation}
has its minimum at a non-zero value
$\varepsilon_* = \Lambda_q/k(q)$ whenever $\Lambda_q \neq 0$.
Thus a parity-symmetric Lagrangian generates an effective vacuum
that spontaneously selects one sign of $\varepsilon$, providing a
geometric analogue of $V\!-\!A$ chirality selection.
This is analyzed in Sec.~\ref{sec:phase2} and Appendix~D.

\paragraph{(3) Critical stiffness scaling and composite states (Phase~3).}
Finally, we consider how the stiffness $k(q)$ scales with $q$ in the
Spinning $SU(2)$ Handle Ansatz.
We show that the leading contribution behaves as
\begin{equation}
  k(q) \;\propto\; m^2 |q|,
\end{equation}
so that in a coarse-grained description the total energy for a charge-$q$
configuration can be written in the form
\begin{equation}
  E(q) \;=\; \alpha |q| - \beta |q|^{2-\gamma},
\end{equation}
with an effective exponent $\gamma = 1$ arising from the geometry of
the handle.
This places the system at a near-critical point where fusion and
fission of handles have comparable energetics.
We discuss how small corrections (e.g.\ quantum effects or additional
interactions) could slightly reduce the effective exponent,
$\gamma_{\mathrm{eff}} \lesssim 1$, allowing certain composite states,
in particular $q=3$, to become energetically favored over their
constituents.
The associated calculations are presented in Sec.~\ref{sec:phase3}
and Appendix~E.

Moreover, the same ansatz yields a stiffness that scales linearly with
$|q|$, placing the system essentially at a critical regime in which
higher-charge composite states become energetically competitive with
separated unit-charge handles once small perturbative corrections are
taken into account.  This is what we will refer to as
\emph{critical binding} in the remainder of the paper.

\subsection{Outline of the paper}
\label{subsec:intro-outline}

The structure of the paper is as follows.
In Sec.~\ref{sec:framework} we review the essentials of teleparallel
gravity and the Nieh--Yan invariant, and we define the topological
charge associated with a single microscopic handle.
Section~\ref{sec:ansatz} introduces the Spinning $SU(2)$ Handle Ansatz
for the tetrad on $M \simeq \mathbb{R} \times S^1 \times S^2$ and
summarizes the resulting torsion and Nieh--Yan density.
In Sec.~\ref{sec:phase1} (Phase~1) we derive the effective
one-dimensional energy functional for the handle radius and show the
existence of a stable radius with $E_{\mathrm{TEGR}} \propto |q|$.
In Sec.~\ref{sec:phase2} (Phase~2) we analyze the precession mode,
its stiffness, and the Nieh--Yan induced linear term, leading to
spontaneous parity breaking.
Section~\ref{sec:phase3} (Phase~3) is devoted to the scaling of
the stiffness with $q$ and to the energetics of composite charge
states.
In Sec.~\ref{sec:discussion} we summarize the logical chain from
geometry to topology to effective ``matter'' properties, and we
comment on possible connections to baryon-like and meson-like
configurations.
Section~\ref{sec:conclusion} contains our conclusions.
Several technical derivations and full expressions for the tetrad
and torsion are collected in the appendices.

\section{Theoretical framework}
\label{sec:framework}

In this section we briefly review the teleparallel formulation of
gravity used in this work, introduce the Nieh--Yan invariant, and
define the topological charge associated with a microscopic handle.
We follow standard conventions in teleparallel gravity, as reviewed
for instance in Aldrovandi and Pereira's textbook~\cite{Aldrovandi:Teleparallel}.

\subsection{TEGR and Weitzenb\"ock geometry}
\label{subsec:TEGR}

The fundamental dynamical variable of teleparallel gravity is a
tetrad (vierbein) field $e^a{}_\mu$ on a four-dimensional manifold
$M$, where Greek indices $\mu,\nu,\dots$ label spacetime coordinates
and Latin indices $a,b,\dots$ label an orthonormal frame.
The spacetime metric is given by
\begin{equation}
  g_{\mu\nu} \;=\; \eta_{ab} e^a{}_\mu e^b{}_\nu,
\end{equation}
where $\eta_{ab} = \mathrm{diag}(-1,+1,+1,+1)$.
The determinant of the tetrad is denoted
$e = \det(e^a{}_\mu) = \sqrt{-\det g_{\mu\nu}}$.

In TEGR one introduces a connection, the Weitzenböck connection,
which is metric-compatible and curvature-free but has non-vanishing
torsion.
In a pure-tetrad formulation this connection can be written as
\begin{equation}
  \Gamma^\rho_{\ \mu\nu}
  \;=\; e_a{}^\rho\, \partial_\nu e^a{}_\mu,
\end{equation}
so that its torsion tensor is
\begin{equation}
  T^\rho_{\ \mu\nu}
  \;=\;
  \Gamma^\rho_{\ \nu\mu} - \Gamma^\rho_{\ \mu\nu}
  \;=\;
  e_a{}^\rho(\partial_\mu e^a{}_\nu - \partial_\nu e^a{}_\mu).
\end{equation}
The curvature built from $\Gamma^\rho_{\ \mu\nu}$ vanishes identically,
$R^\rho_{\ \sigma\mu\nu}(\Gamma) \equiv 0$,
so that all gravitational degrees of freedom are encoded in the
torsion.

The TEGR Lagrangian density is constructed from a scalar $\mathbb{T}$
quadratic in the torsion,
\begin{equation}
  \mathbb{T}
  \;=\;
  \frac{1}{4} T^\rho_{\ \mu\nu} T_\rho^{\ \mu\nu}
  + \frac{1}{2} T^\rho_{\ \mu\nu} T^{\nu\mu}{}_\rho
  - T^\rho_{\ \mu\rho} T^{\sigma\mu}{}_\sigma,
\end{equation}
and the gravitational action reads
\begin{equation}
  S_{\mathrm{TEGR}}
  \;=\;
  \frac{1}{2\kappa^2}
  \int_M d^4x\, e\, \mathbb{T},
  \qquad
  \kappa^2 = 8\pi G.
\end{equation}
Up to a total divergence, this action is equivalent to the
Einstein--Hilbert action built from the Levi-Civita connection of
$g_{\mu\nu}$, and thus reproduces the same classical field equations.
In what follows we work in units where $\kappa^2 = 1$ when convenient,
and we leave the inclusion of matter fields implicit, focusing on
the gravitational/torsional sector that is relevant for the
microscopic-handle dynamics.

\subsection{Nieh--Yan invariant and effective Wess--Zumino term}
\label{subsec:NY}

In a first-order (tetrad and spin connection) formulation of gravity
with torsion, one may define the Nieh--Yan four-form
\begin{equation}
  \mathcal{N}
  \;=\;
  T^a \wedge T_a - R_{ab} \wedge e^a \wedge e^b,
\end{equation}
where $T^a$ is the torsion two-form, $R_{ab}$ is the curvature two-form
of the spin connection, and $e^a$ is the coframe one-form.
In the Weitzenböck geometry used in TEGR the curvature vanishes,
$R_{ab} \equiv 0$, so the Nieh--Yan form simplifies to
\begin{equation}
  \mathcal{N}
  \;=\;
  T^a \wedge T_a
  \;=\;
  d\bigl(e^a \wedge T_a\bigr),
\end{equation}
i.e.\ it is an exact form and hence a total divergence locally.
However, on manifolds with non-trivial topology or with boundaries
(such as our microscopic handles and the regions gluing them to the
rest of spacetime), its integral can be non-zero and topologically
quantized, in close analogy with the way instanton densities contribute
to anomalies in gauge theories.

When coupled to chiral fermions, the Nieh--Yan term can
contribute to the chiral anomaly.
Chandia and Zanelli showed that, in spaces with torsion, the divergence
of the axial current receives an additional contribution proportional
to $\mathcal{N}$, so that the anomaly equation schematically takes the
form
\begin{equation}
  \partial_\mu J_5^\mu
  \;\sim\;
  \frac{1}{16\pi^2}
  \bigl(\mathrm{tr}\, F\wedge F
        + \cdots
        + \lambda_{\mathrm{NY}}\, \mathcal{N}\bigr),
\end{equation}
where $F$ denotes the gauge field strength and
$\lambda_{\mathrm{NY}}$ is a dimensionful coefficient depending on
the ultraviolet regularization~\cite{Chandia:1997hu}.
In an effective low-energy description, such an anomaly can be
encoded by adding a Wess--Zumino--Witten-type term to the action
of an $SU(2)$-valued field $U(x)$, with a quantized level
$k_{\mathrm{WZ}}$~\cite{Witten:1983tw}.

Anomaly-matching arguments therefore suggest that, in an appropriate
effective description, the integral of $\mathcal{N}$ over a
four-dimensional region may control the coefficient of such a
topological term.  In this paper we do not attempt a full derivation
of the effective action starting from a microscopic fermion and
gauge-field content. Instead, we adopt the following
\emph{working hypothesis}:
\begin{quote}
  For a suitably defined $SU(2)$-valued field associated with the
  local frame of a microscopic handle, the integral of the Nieh--Yan
  density over that handle contributes to an induced Wess--Zumino--
  Witten-type topological term in the low-energy effective action,
  with a coefficient that receives a contribution proportional to the
  integer-valued Nieh--Yan charge $q$.
\end{quote}
In the explicit handle model constructed below we will in fact find
that the effective energy contains a term linear in $q$ that plays
a role analogous to a WZW level, and this linear dependence on $q$
is the only feature of the induced topological term that we will
actually use in our subsequent analysis.

Under this hypothesis, a handle with odd $q$ is expected to behave
as a fermionic defect, while one with even $q$ is bosonic,
at least at the level of an effective Skyrme-like description.
The precise proportionality factor and the matching to a concrete
microscopic model are left for future work; here we only need the
existence of an integer $q$ that labels topological sectors and
enters linearly in the effective action.

\subsection{Topological charge of a microscopic handle}
\label{subsec:top-charge}

We now define the topological charge associated with a single
microscopic handle.
Let $M \simeq \mathbb{R}_t \times S^1_\psi \times S^2_{(\theta,\phi)}$
denote a four-dimensional region containing one handle, with
coordinates $x^\mu = (t,\psi,\theta,\phi)$.
We assume that the tetrad is such that the Nieh--Yan form
$\mathcal{N}$ is integrable over $M$ and that the fields fall off
sufficiently fast outside the handle so that boundary contributions
from infinity can be neglected.

We define the Nieh--Yan charge $q$ of the handle by
\begin{equation}
  q
  \;=\;
  \frac{1}{\mathcal{N}_0}
  \int_M \mathcal{N},
  \label{eq:q-def}
\end{equation}
where $\mathcal{N}_0$ is a normalization constant chosen so that
$q \in \mathbb{Z}$ for the class of configurations considered.
In a microscopic UV completion, the natural quantization unit that
guarantees $q \in \mathbb{Z}$ for topologically non-trivial
configurations is
\begin{equation}
  \mathcal{N}_0 = 32\pi^2,
\end{equation}
in exact analogy with the instanton charge in Yang--Mills
theory~\cite{Chandia:1997hu}.
In a fully microscopic theory this quantization would follow from the
global properties of the tetrad and spin connection, but for the
present purposes we simply assume that $q$ takes integer values.

The decomposition $\mathcal{N} = d(e^a \wedge T_a)$ implies that
$q$ is determined by the flux of $e^a \wedge T_a$ through the
boundary of $M$, which in our setting consists of a union of
three-dimensional hypersurfaces connecting the handle to the rest
of spacetime.
This makes $q$ a natural measure of the amount of torsional
``flux'' threading the handle.
In the Spinning $SU(2)$ Handle Ansatz introduced in
Sec.~\ref{sec:ansatz}, the explicit form of $\mathcal{N}$ will be
seen to depend on the global twist and spin of the tetrad around
the handle, so that $q$ is directly related to the corresponding
winding data.
Since the detailed structure is somewhat involved, we postpone its
full expression to Sec.~\ref{sec:ansatz} and to the appendices;
for now, Eq.~\eqref{eq:q-def} serves as our definition of the
topological charge labeling microscopic-handle configurations.

In summary, the theoretical framework we adopt consists of:
(i) TEGR as the gravitational theory, with torsion encoded in a
tetrad on $M$,
(ii) the Nieh--Yan density as a total derivative that nevertheless
enters the chiral anomaly and can induce an effective WZW-like
topological term, and
(iii) an integer-valued Nieh--Yan charge $q$ defined by the integral
of $\mathcal{N}$ over a microscopic handle.
In Sec.~\ref{sec:ansatz} we specify the Spinning $SU(2)$ Handle Ansatz
that realizes these ingredients in an explicit tetrad configuration.

\section{Spinning $SU(2)$ Handle Ansatz}
\label{sec:ansatz}

In this section we specify the microscopic geometry of a single handle and
introduce the \emph{Spinning $SU(2)$ Handle Ansatz} that will be used
throughout Phases~1--3. The ansatz encodes three charges $(q,m,\omega)$
and a radius profile $r(\psi)$ in a tetrad field $e^a{}_\mu$ defined on a
fixed handle topology.

\subsection{Geometry of a single handle}
\label{subsec:ansatz-geometry}

Locally we model a single microscopic handle as a product manifold
\begin{equation}
  M \simeq \mathbb{R}_t \times S^1_\psi \times S^2_{(\theta,\phi)},
\end{equation}
with coordinates
\begin{equation}
  x^\mu = (t,\psi,\theta,\phi),
\end{equation}
where $t$ denotes physical time, $\psi\in[0,2\pi)$ is an angular coordinate
along the handle axis ($S^1$), and $(\theta,\phi)$ are polar and azimuthal
angles on the cross--section $S^2$.

We assume that the handle is much smaller than any macroscopic curvature
scale of the ambient universe, so that background fields may be treated
as approximately constant over $M$.

As in Phase~1, we keep a single geometric degree of freedom for the
shape of the handle: the physical radius $r(\psi)$ of the two--sphere
cross--section as a function of the axial coordinate~$\psi$.
A rough form of the line element is then
\begin{equation}
  ds^2 \sim -dt^2 + d\psi^2
            + r(\psi)^2\bigl(d\theta^2 + \sin^2\theta\,d\phi^2\bigr),
  \label{eq:ansatz-metric-rough}
\end{equation}
while the exact expression will be encoded in the tetrad defined below.
In particular, the $q$--dependent terms that implement the torsion flux
are captured at the tetrad level and are not shown explicitly in
\eqref{eq:ansatz-metric-rough}.

\subsection{Tetrad construction and parameters}
\label{subsec:ansatz-tetrad}

We work in the Weitzenb\"ock gauge, so that the spin connection vanishes
and all gravitational information is carried by the tetrad field
$e^a{}_\mu$.

The handle ansatz is constructed in two steps:
we first choose a convenient \emph{reference tetrad} $\tilde e^a{}_\mu$
with a monopole--like torsion flux, and then act with
a time-- and angle--dependent spatial rotation derived from an
$SU(2)$ group element.

\subsubsection*{Reference tetrad and monopole charge $q$}

For definiteness we take the reference tetrad one--forms to be
\begin{align}
  \tilde e^0 &= dt, \label{eq:ref-tetrad-0}\\
  \tilde e^1 &= d\psi, \label{eq:ref-tetrad-1}\\
  \tilde e^2 &= r(\psi)\,d\theta, \label{eq:ref-tetrad-2}\\
  \tilde e^3 &= r(\psi)\,\sin\theta\,
               \bigl(d\phi + q(1-\cos\theta)\,d\psi\bigr).
               \label{eq:ref-tetrad-3}
\end{align}
The resulting metric is approximately of the form
\eqref{eq:ansatz-metric-rough}; the $q$--dependent mixing term resides
in $\tilde e^3$ and will play a role only in the torsion sector.

We define $q\in\mathbb{Z}$ as an \emph{integer monopole charge} associated
with the torsion flux through each cross--section $S^2_\psi$ at fixed
$(t,\psi)$.
Concretely, one may impose a normalisation condition of the form
\begin{equation}
  \frac{1}{4\pi}\int_{S^2_\psi} \!\!\!\star T^1 = q,
\end{equation}
where $T^a = d\tilde e^a$ is the torsion 2--form and $\star$ denotes the
Hodge dual built from the induced metric.
In what follows we treat $q$ as a fixed topological input for each handle.

\subsubsection*{$SU(2)$ field and induced rotation}

We introduce an internal $SU(2)$ field
\begin{equation}
  U(t,\psi)
  \;=\;
  \exp\!\left(i\frac{\omega t}{2}\sigma_1\right)\,
  \exp\!\left(i\frac{m\psi}{2}\sigma_3\right),
  \label{eq:SU2-U}
\end{equation}
where $\sigma_i$ are the Pauli matrices.
The first factor implements an intrinsic \emph{spin} in time, while the
second factor implements a \emph{twist} along the handle axis.
We therefore refer to \eqref{eq:SU2-U} as a ``spin--then--twist''
parameterisation.

Via the double cover $SU(2)\to SO(3)$, the field $U(t,\psi)$ defines
a spatial rotation matrix $\Lambda^i{}_j(t,\psi)$ acting on the spatial
tetrad indices $i,j=1,2,3$.
We then define the physical tetrad as
\begin{equation}
  e^0 = \tilde e^0,
  \qquad
  e^i = \Lambda^i{}_j(t,\psi)\,\tilde e^j,
  \quad i=1,2,3.
  \label{eq:phys-tetrad}
\end{equation}
In the explicit implementation used in the detailed calculations,
the rotation factorises as
\begin{equation}
  \Lambda(t,\psi) = R_3(m\psi)\,R_1(\omega t),
  \label{eq:rot-matrix-factor}
\end{equation}
so that the spatial basis is first spun around the local
$\tilde e^1$--axis as time evolves, and then twisted in the
$(\tilde e^2,\tilde e^3)$--plane as one moves along $\psi$.

A key property of this ordering is that the handle axis vector $e^1$
remains \emph{time--independent}: the spin acts around the axis rather
than precessing the axis itself.
This will allow us to add a small precession mode $\varepsilon(t)$
in Phase~2 without destabilising the background configuration.

\subsubsection*{Summary of parameters}

The ansatz introduces the following parameters and fields:
\begin{itemize}
  \item $q\in\mathbb{Z}$ (flux / monopole charge):
    integer--quantised torsion flux through the $S^2$ cross--section.
    It controls the strength of the core repulsion and, via the
    Nieh--Yan/WZW chain, the effective WZW level in a boundary
    Skyrme--like description.
  \item $m\in\mathbb{Z}$ (twist number):
    twist number along the handle axis; as $\psi$ increases from $0$
    to $2\pi$, the internal frame rotates $m$ times in the cross--section.
  \item $\omega\in\mathbb{R}$ (spin):
    intrinsic spin angular frequency of the handle, defined in the
    co--moving frame of the twisted handle.
  \item $r(\psi) > 0$ (radius profile):
    radius of the $S^2$ cross--section as a function of $\psi$.
    In Phase~1 we focus on the equal--radius ansatz $r(\psi)=r_0$,
    while Phases~2 and~3 allow small deformations.
\end{itemize}

In the TEGR action, $q$, $m$, $\omega$ and $r(\psi)$ control,
respectively, the flux--induced core energy, the twist energy,
the spin--related terms, and the elastic energy associated with
variations of the radius.
Their scaling behaviours will be analysed in detail in
Sec.~\ref{sec:phase1} and subsequent sections.

\subsection{Nieh--Yan density as an exact form}
\label{subsec:ansatz-NY}

In the Weitzenb\"ock geometry adopted here, the Nieh--Yan 4--form
simplifies to
\begin{equation}
  \mathcal N = T^a\wedge T_a = d(e^a\wedge T_a),
  \label{eq:NY-exact}
\end{equation}
and behaves as a torsion--built topological invariant.

For the Spinning $SU(2)$ Handle Ansatz, an explicit calculation of the
torsion 2--forms $T^a = de^a$ yields a Nieh--Yan density that, up to an
overall normalisation fixed by the choice of $q$, takes the form
\begin{align}
  \mathcal N
  &\simeq
  q\,\omega\,m\,
  \partial_\psi\bigl(r(\psi)^2\bigr)\,
  dt\wedge d\psi\wedge
  \sin\theta\,d\theta\wedge d\phi
  \nonumber\\
  &=\;
  q\,\omega\,m\,
  d\!\bigl(r(\psi)^2\bigr)\wedge dt\wedge d\Omega_2,
  \label{eq:NY-total-derivative}
\end{align}
where $d\Omega_2 = \sin\theta\,d\theta\wedge d\phi$ is the area form
on the unit $2$--sphere, and $\simeq$ denotes equality up to a
numerical normalisation that will be fixed once and for all by the
flux quantisation condition for $q$.

This total--derivative structure has two important consequences.

\paragraph{(i) Boundary sensitivity.}
Because $\mathcal N$ is an exact form in the $\psi$--direction,
the Nieh--Yan contribution of a single handle depends only on the
boundary values of $r^2$ at the ends of the handle (or, more precisely,
at the joints where the handle is glued to the ambient manifold):
\begin{equation}
  \int_M \mathcal N
  \;\propto\;
  q\,\omega\,m\,
  \bigl[r(\psi)^2\bigr]_{\psi\text{--endpoints}}.
  \label{eq:NY-boundary}
\end{equation}
For a microscopic handle that is nearly cylindrical in its interior,
the contribution is thus localised near the joints.

\paragraph{(ii) Effective WZW level.}
We emphasise that $q$ is defined independently, as the integer monopole
charge of the torsion flux in the reference tetrad, and is \emph{not}
redefined via the Nieh--Yan integral.

Instead, we interpret the Nieh--Yan integral as a measure of an
effective WZW--like level seen by a boundary Skyrme field.
For a single handle, the scaling behaviour implied by
\eqref{eq:NY-total-derivative} can be summarised as
\begin{equation}
  \int_M \mathcal N
  \;\propto\;
  q\,\omega\,m\,
  \Delta\!\bigl(r^2\bigr),
  \label{eq:NY-WZW-scaling}
\end{equation}
where $\Delta(r^2)$ denotes the jump of $r^2$ between the two ends of
the handle.
In the multi--handle ensemble considered in later sections, the sum of
such contributions determines the effective WZW level that appears in
the boundary chiral action.
This underlies the anomaly--matching picture developed in
Sec.~\ref{sec:boundary}.

The detailed derivation of \eqref{eq:NY-total-derivative} from the
tetrad \eqref{eq:ref-tetrad-0}--\eqref{eq:ref-tetrad-3} and
\eqref{eq:phys-tetrad}--\eqref{eq:rot-matrix-factor} is straightforward
but lengthy; it is relegated to Appendix~B.

\subsection{Static versus spinning ans\"atze}
\label{subsec:ansatz-static-vs-spinning}

It is instructive to contrast the spinning ansatz above with more
naive alternatives.

If one sets $\omega=0$ and keeps only a static twist
$U(\psi)=\exp(i m\psi\,\sigma_3/2)$, the Nieh--Yan density
\eqref{eq:NY-total-derivative} loses the cross term proportional to
$\omega m$.
In such a purely twisted configuration the handle still carries a
flux $q$, but the Nieh--Yan density no longer exhibits the simple
total--derivative structure that makes the identification of an
effective WZW level transparent.

Conversely, if one tries to encode precession by letting the handle
axis itself rotate in time with respect to a fixed laboratory frame,
the tetrad becomes time--dependent already at the level of the axis
vector $e^1$.
This leads to an unnecessarily complicated TEGR energy functional and
obscures the separation between
\begin{itemize}
  \item the \emph{background} spinning handle
        (characterised by $q,m,\omega,r(\psi)$), and
  \item the \emph{small precession mode} $\varepsilon(t)$ that we
        introduce in Phase~2.
\end{itemize}

The present Spinning $SU(2)$ Handle Ansatz avoids both problems:
\begin{itemize}
  \item the axis $e^1$ is time--independent, so that a small tilt
        $\varepsilon(t)$ can be added cleanly on top of the spinning
        background, and
  \item the Nieh--Yan density acquires the exact--form structure
        \eqref{eq:NY-total-derivative}, which allows us to treat
        $\int_M\mathcal N$ as a bona fide contribution to an effective
        WZW level in the boundary theory.
\end{itemize}

For these reasons we adopt the spinning ansatz
\eqref{eq:ref-tetrad-0}--\eqref{eq:ref-tetrad-3} and
\eqref{eq:SU2-U}--\eqref{eq:rot-matrix-factor} as our
\emph{canonical microscopic geometry} for a single handle, and relegate
static variants and precessing laboratory--frame configurations to
Appendix~F.


\section{Phase 1: Stability and energy scaling}
\label{sec:phase1}

In this section we place the Spinning $SU(2)$ handle ansatz of
Sec.~\ref{sec:ansatz} into the TEGR action of Sec.~\ref{sec:TEGR}, and
reduce the dynamics to the single radial degree of freedom $r(\psi)$ in a
one-dimensional mini-superspace.  We then show that pure TEGR already
admits a classically stable equal-radius handle solution with a radius
\begin{equation}
  r_0 \;\propto\; |q|^{1/2},
  \qquad
  \text{(for fixed non-zero twist number $m$)},
  \label{eq:r0-scaling-intro}
\end{equation}
and that the corresponding TEGR energy scales linearly with the absolute
value of the monopole charge $q$,
\begin{equation}
  E_{\text{TEGR}}(q)
  \;\propto\; |q|,
  \qquad
  \text{(again with fixed non-zero $m$)}.
\end{equation}
These results provide the $P$--even geometric background for the
parity-violating effects discussed in Phase~2 and the binding analysis
in Phase~3.

%------------------------------------------------------------
\subsection{Effective mini-superspace energy functional}
\label{subsec:phase1-effective}

We work in pure TEGR,
\begin{equation}
  S_{\text{TEGR}}
  \;=\;
  \frac{1}{2\kappa^2}
  \int d^4x\, e\, \mathbb{T},
\end{equation}
with Weitzenb\"ock connection and torsion scalar $\mathbb{T}$ as defined in
Sec.~\ref{sec:TEGR}.  On the local product manifold
$M \simeq \mathbb{R}_t \times S^1_\psi \times S^2$, the Spinning
$SU(2)$ handle ansatz of Sec.~\ref{sec:ansatz} reduces the geometric
degrees of freedom to a single radial function $r(\psi)$ measuring
the isotropic radius of the $S^2$ cross-section at each point
along the handle axis $S^1_\psi$.

After inserting the ansatz into $S_{\text{TEGR}}$, averaging over time,
and integrating over the $S^2$ cross-section, one obtains an effective
one-dimensional energy functional for $r(\psi)$,
\begin{equation}
  E[r]
  \;=\;
  \int_0^{L_\psi} d\psi\,
  \Big[
    A\,(r')^2 + V(r)
  \Big],
  \qquad
  r' := \frac{dr}{d\psi},
  \label{eq:effective-energy-functional}
\end{equation}
where $L_\psi$ denotes the coordinate length of the handle along the
$\psi$--direction and $A>0$ is a positive coefficient determined by the
TEGR normalisation and the detailed tetrad choice.
The coefficient $A$ arises from axial components of the torsion scalar
(schematically from terms of the form
$T^{\psi}{}_{r\psi} T_\psi{}^{r\psi}$, etc.) and is expected to be of
Planckian magnitude in a UV-complete theory.

The effective potential $V(r)$ for an equal-radius handle takes the form
\begin{equation}
  V(r)
  \;=\;
  \frac{B\,q^2}{r^2} + C\,m^2 r^2,
  \qquad B>0,\;C>0,
  \label{eq:effective-potential}
\end{equation}
where
\begin{itemize}
  \item $q \in \mathbb{Z}$ is the integer monopole charge associated with
        the torsion flux through the $S^2$ cross-section, as defined in
        Sec.~\ref{sec:ansatz};
  \item $m \in \mathbb{Z}$ is the twist number of the $SU(2)$ rotation
        along the handle axis;
  \item $B$ and $C$ are positive constants set by the TEGR coupling
        and the detailed geometry of the ansatz.
\end{itemize}
The first term, $\propto q^2/r^2$, originates from the flux contribution
to the torsion scalar and behaves as a repulsive ``core'' potential that
diverges for $r \to 0$.  The second term, $\propto m^2 r^2$, arises from
the uniform twist along $S^1_\psi$ and plays the rôle of a tension or
centrifugal contribution that diverges for $r \to \infty$.

Spin (self-rotation) $\omega$ does contribute to the torsion scalar, but
in the co-rotating frame implicit in the ansatz its $r$--dependent
contributions cancel between kinetic and centrifugal terms for static,
equal-radius configurations.  As a result, $\omega$ does not appear in
$V(r)$, and its rôle is deferred to the Nieh-Yan sector and the
precession dynamics discussed in later sections.

%------------------------------------------------------------
\subsection{Classical stability of the equal-radius handle}
\label{subsec:phase1-stability}

We now specialise to static equal-radius configurations
\begin{equation}
  r(\psi) = r_0 = \text{const.},
\end{equation}
for which the gradient term in
\eqref{eq:effective-energy-functional} vanishes and the energy reduces
to
\begin{equation}
  E[r_0]
  \;=\;
  \int_0^{L_\psi} d\psi\, V(r_0)
  \;=\;
  L_\psi\, V(r_0).
  \label{eq:E-equal-radius}
\end{equation}
The classical equilibrium radius is thus determined by the extremum
condition
\begin{equation}
  \frac{dV}{dr}\Big|_{r=r_0} = 0.
\end{equation}
Using \eqref{eq:effective-potential},
\begin{equation}
  \frac{dV}{dr}
  \;=\;
  -\,\frac{2B q^2}{r^3} + 2C m^2 r,
\end{equation}
so that the extremum satisfies
\begin{equation}
  C m^2 r_0^4 = B q^2
  \qquad\Longrightarrow\qquad
  r_0^4 = \frac{B q^2}{C m^2}.
  \label{eq:r0-fourth}
\end{equation}
For $q\neq 0$ and $m\neq 0$ this yields a finite, nonzero equilibrium
radius
\begin{equation}
  r_0
  \;=\;
  \left(\frac{B}{C}\right)^{1/4}
  \frac{|q|^{1/2}}{|m|^{1/2}},
  \label{eq:r0-explicit}
\end{equation}
up to a theory-dependent numerical factor $(B/C)^{1/4}$.  In particular,
for a fixed non-zero twist number $m$ the radius scales as
\begin{equation}
  r_0 \;\propto\; |q|^{1/2},
  \qquad
  (m \text{ fixed}),
\end{equation}
in agreement with the scaling quoted in
Eq.~\eqref{eq:r0-scaling-intro}.

To check that this extremum is a minimum, we compute the second
derivative of $V(r)$,
\begin{equation}
  \frac{d^2V}{dr^2}
  \;=\;
  \frac{6B q^2}{r^4} + 2C m^2.
\end{equation}
Evaluating at $r=r_0$ and using \eqref{eq:r0-fourth},
\begin{equation}
  \frac{d^2V}{dr^2}\Big|_{r=r_0}
  \;=\;
  \frac{6B q^2}{r_0^4} + 2C m^2
  \;=\;
  6C m^2 + 2C m^2
  \;=\;
  8C m^2 \;>\; 0,
\end{equation}
so the equal-radius configuration is indeed a local minimum of the
effective potential for any $q\neq 0$ and $m\neq 0$.

Small fluctuations $\delta r(\psi)$ around $r_0$ can be analysed by
expanding \eqref{eq:effective-energy-functional} to quadratic order,
\begin{equation}
  r(\psi) = r_0 + \delta r(\psi),
\end{equation}
which yields a Sturm-Liouville problem with a positive mass term
proportional to $d^2V/dr^2|_{r_0}$ and a gradient term controlled by
the coefficient $A$.  In the regime of interest for this geometric mechanism, $A$ is expected
to be of Planckian order in natural units, so that non-uniform Fourier
modes of $\delta r$ are heavy and strongly suppressed.  Consequently, in
the semiclassical regime relevant to our handle model the equal-radius configuration
$r(\psi)=r_0$ provides an excellent approximation to the true vacuum,
with radial oscillations confined to sub-Planckian amplitudes.
(A more detailed estimate of $A$ and the fluctuation spectrum is
deferred to Appendix~\ref{app:phase1}.)

%------------------------------------------------------------
\subsection{On-shell TEGR energy and linear scaling in $|q|$}
\label{subsec:phase1-energy-scaling}

We now evaluate the TEGR contribution to the energy on the equal-radius
solution $r(\psi)=r_0$ determined above.  Substituting
\eqref{eq:r0-fourth} into \eqref{eq:effective-potential}, we obtain
\begin{align}
  V(r_0)
  &= \frac{B q^2}{r_0^2} + C m^2 r_0^2
  \nonumber\\[2pt]
  &= \frac{B q^2}{\bigl(\tfrac{B}{C}\bigr)^{1/2} |q/m|}
     + C m^2 \bigl(\tfrac{B}{C}\bigr)^{1/2} \Bigl|\frac{q}{m}\Bigr|
  \nonumber\\[2pt]
  &= \sqrt{BC}\,|m q| + \sqrt{BC}\,|m q|
  \nonumber\\[2pt]
  &= 2\sqrt{BC}\,|m q|.
  \label{eq:V-r0}
\end{align}
The corresponding TEGR energy is
\begin{equation}
  E_{\text{TEGR}}(q,m)
  \;=\;
  L_\psi\, V(r_0)
  \;=\;
  2L_\psi \sqrt{BC}\,|m q|.
  \label{eq:TEGR-energy-on-shell}
\end{equation}
For fixed twist number $m\neq 0$ and handle length $L_\psi$ this implies
the linear scaling
\begin{equation}
  E_{\text{TEGR}}(q)
  \;\equiv\; E_{\text{TEGR}}(q,m_{\rm fixed})
  \;\approx\;
  \alpha\,|q|,
  \qquad
  \alpha := 2L_\psi \sqrt{BC}\,|m| > 0.
  \label{eq:E-TEGR-linear}
\end{equation}
Thus, within the Spinning $SU(2)$ handle ansatz, pure TEGR behaves as a
\emph{geometric mass term} that is linear in the absolute value of the
monopole charge $q$.  The twist number $m\neq 0$ is required for the
existence of a stable minimum and sets the overall energy scale, but it
does not modify the linear dependence on $|q|$.

In particular, the TEGR energy is insensitive to the sign of $q$; it is
even under parity and charge conjugation, and does not distinguish
between $q$ and $-q$.  The sign-sensitivity and chirality selection that
are essential for this framework therefore cannot come from the TEGR
sector alone.  They enter through the Nieh--Yan term and its effective
boundary coupling to the precession mode, which we analyse in Phase~2.

\section{Phase 2: Nieh--Yan boundary term and the precession mode}
\label{sec:phase2}

In Phase~1 (Sec.~\ref{sec:phase1}) we showed that, within the Spinning
$SU(2)$ handle ansatz, pure TEGR admits a classically stable equal--radius
configuration $r(\psi)=r_0$ whose energy scales as
$E_{\text{TEGR}}(q)\propto |q|$ for fixed non-zero twist number $m$.
This background is parity--even: it does not distinguish between $q$ and
$-q$ and does not select any preferred handedness for the handle.

In this section we switch on the Nieh--Yan term and introduce a slow
precession mode that tilts the handle axis by a small time-dependent
angle $\varepsilon(t)$.  We show that the combined effect of TEGR and
Nieh--Yan is an effective potential of the form
\begin{equation}
  V_{\text{eff}}(\varepsilon; q)
  \;\simeq\;
  E_{\text{TEGR}}(q)
  + \frac{1}{2}k(q)\,\varepsilon^2
  + \Lambda_q\,\varepsilon
  + \mathcal{O}(\varepsilon^4),
  \label{eq:Veff-intro}
\end{equation}
where $k(q)>0$ is a positive function of $q$ and $\Lambda_q$ is a
coefficient linear in $q$.  In the sign convention adopted here we write
\begin{equation}
  \Lambda_q \;=\; -\,\gamma\,q,
  \qquad \gamma>0 \;\;\text{(for fixed $m,\omega,\theta_{\text{NY}},\Delta r^2$)},
  \label{eq:Lambdaq-sign}
\end{equation}
so that $\Lambda_{-q}=-\Lambda_q$ and the sign of $\Lambda_q$ is opposite
to the sign of $q$.

The minimum of \eqref{eq:Veff-intro} occurs at
\begin{equation}
  \varepsilon_*(q)
  \;\simeq\;
  -\,\frac{\Lambda_q}{k(q)}
  \;=\;
  \frac{\gamma}{k(q)}\,q,
  \label{eq:epsilon-star-intro}
\end{equation}
so that, for $\Lambda_q\neq 0$, the true ground state is slightly tilted
($\varepsilon_*\neq 0$) and its sign is tied to the sign of $q$.  This
provides a simple geometric mechanism for dynamical chirality selection:
handles with $q>0$ and $q<0$ precess in opposite directions.

Throughout this section we work in the regime
\begin{equation}
  |\varepsilon(t)| \ll 1,
  \qquad
  \Bigl|\frac{d\varepsilon}{dt}\Bigr|
  \;\text{small},
\end{equation}
so that an expansion in powers of $\varepsilon$ and its time derivatives
is consistent.  Large precession angles require a separate non-perturbative
treatment and lie beyond the scope of the present paper.

%------------------------------------------------------------
\subsection{Introducing the precession mode}
\label{subsec:phase2-precession}

We start from the equal--radius background of Phase~1,
\begin{equation}
  r(\psi) = r_0,
\end{equation}
with the Spinning $SU(2)$ handle ansatz of Sec.~\ref{sec:ansatz},
schematically
\begin{equation}
  U(\psi,t)
  \;=\;
  \text{Twist}(m\psi)\,\text{Spin}(\omega t),
\end{equation}
and the associated tetrad $e^a{}_\mu$ that realises a wormhole--like handle
of radius $r_0$ and twist number $m$.

To describe a slow precession of the handle axis, we introduce a small,
spatially uniform rotation in the internal $1$--$2$ plane,
\begin{equation}
  e^1 + i e^2
  \;\longrightarrow\;
  e^{i\varepsilon(t)} (e^1 + i e^2),
  \qquad
  |\varepsilon(t)| \ll 1,
  \label{eq:precession-rotation}
\end{equation}
while keeping $r(\psi)=r_0$ fixed.
Concretely, at the level of the $SU(2)$ field this precession mode can
be implemented by an additional factor
\begin{equation}
  U_{\text{prec}}(t)
  \;=\;
  \exp\!\left(\frac{i}{2}\,\varepsilon(t)\,\sigma_2\right),
  \label{eq:U-prec}
\end{equation}
acting on the left of the background configuration, so that the full
$SU(2)$ field becomes
\begin{equation}
  U(\psi,t)
  \;=\;
  U_{\text{prec}}(t)\,
  \exp\!\left(i\frac{m\psi}{2}\sigma_3\right)
  \exp\!\left(i\frac{\omega t}{2}\sigma_1\right).
  \label{eq:U-with-prec}
\end{equation}
Via the $SU(2)\to SO(3)$ map, this corresponds to a time--dependent
rotation of the spatial triad about the internal 2--axis, i.e. a slow
tilt of the handle axis towards the $e^2$--direction.

We treat $\varepsilon(t)$ as a collective coordinate parametrising the
slow precession of the entire handle.  Inserting the precessing tetrad
into the TEGR action and expanding around $\varepsilon=0$, one obtains an
effective contribution to the energy of the form
\begin{equation}
  E_{\text{TEGR}}(q,\varepsilon)
  \;=\;
  E_{\text{TEGR}}(q)
  + \frac{1}{2}k(q)\,\varepsilon^2
  + \mathcal{O}(\varepsilon^4),
  \label{eq:V-TEGR-eps}
\end{equation}
where:

\begin{itemize}
  \item $E_{\text{TEGR}}(q)$ is the equal--radius energy derived in
        Sec.~\ref{sec:phase1}, scaling as $\propto |q|$ for fixed $m$;
  \item $k(q)>0$ is an effective \emph{tilt stiffness} generated by the
        TEGR sector.
\end{itemize}

Because the TEGR action is parity--even and invariant under
$\varepsilon \to -\varepsilon$, only even powers of $\varepsilon$ appear,
and the leading nontrivial term is quadratic.  The positivity of $k(q)$
ensures that, in the absence of the Nieh--Yan term, the configuration
$\varepsilon=0$ is a local minimum.

At this stage we do not need the explicit expression for $k(q)$.  We only
assume that $k(q)$ is positive for the range of $q$ of interest.  In
Phase~3 we will show that, within the present ansatz, $k(q)$ scales
linearly with $|q|$,
\begin{equation}
  k(q) \;\propto\; |q|,
\end{equation}
which will play a key rôle in balancing the TEGR energy against the
Nieh--Yan contribution.

%------------------------------------------------------------
\subsection{Nieh--Yan term as an effective boundary coupling}
\label{subsec:phase2-NY}

In Sec.~\ref{sec:TEGR} we introduced the Nieh--Yan 4--form
\begin{equation}
  \mathcal{N}
  \;=\;
  d\bigl(e^a \wedge T_a\bigr)
  \;=\;
  T^a \wedge T_a - e^a \wedge e^b \wedge R_{ab}[\omega],
\end{equation}
and the corresponding topological term
\begin{equation}
  S_{\text{NY}}
  \;=\;
  \theta_{\text{NY}}
  \int_M \mathcal{N},
  \label{eq:S-NY-def}
\end{equation}
with dimensionless coupling $\theta_{\text{NY}}$.  In the teleparallel
setting $R_{ab}[\omega]=0$, so that $\mathcal{N}$ is exact,
\begin{equation}
  \mathcal{N}
  \;=\;
  d\bigl(e^a \wedge T_a\bigr),
\end{equation}
and on a manifold with (effective) boundary $\partial M$ one has
\begin{equation}
  S_{\text{NY}}
  \;=\;
  \theta_{\text{NY}}
  \int_{\partial M} e^a \wedge T_a.
  \label{eq:S-NY-boundary}
\end{equation}

For the Spinning $SU(2)$ handle we take
\begin{equation}
  M \simeq \mathbb{R}_t \times S^1_\psi \times S^2
\end{equation}
and model the two ends of the handle in the $\psi$--direction (and/or the
junctions to the surrounding bulk) as an effective boundary $\partial M$.
The detailed matching to the bulk geometry is not needed here; all that
matters is that the precessing tetrad induces a nontrivial value of
$e^a \wedge T_a$ on $\partial M$.

For the non–precessing Spinning $SU(2)$ handle, Sec.~\ref{sec:ansatz}
showed that the Nieh--Yan density takes, up to an overall normalisation,
the exact total–derivative form
\begin{equation}
  \mathcal{N}
  \;\simeq\;
  q\,\omega\,m\,
  d\!\bigl(r(\psi)^2\bigr)\wedge dt\wedge d\Omega_2,
  \label{eq:NY-total-derivative-phase2}
\end{equation}
where $d\Omega_2$ is the area form on the unit 2–sphere.  Integrating
over $\psi$ and $S^2$ one obtains a contribution proportional to the
difference of $r^2$ at the two endpoints of the handle,
\begin{equation}
  \int_M \mathcal{N}
  \;\propto\;
  q\,\omega\,m\,
  \bigl[r(\psi)^2\bigr]_{\psi\text{--endpoints}}.
  \label{eq:NY-boundary-phase2}
\end{equation}

The precession mode $\varepsilon(t)$ modifies the matching between the
handle and the bulk at the endpoints and effectively shifts the boundary
values of $r^2$ by an amount proportional to $\varepsilon(t)$:
schematically,
\begin{equation}
  \Delta\!\bigl(r^2\bigr)
  \;\longrightarrow\;
  \Delta\!\bigl(r^2\bigr)
  + \beta\,\varepsilon(t),
\end{equation}
with some positive constant $\beta$ determined by the geometry of the
junction.  Inserting this into \eqref{eq:NY-boundary-phase2} and then
into \eqref{eq:S-NY-def} yields an effective contribution to the action
of the form
\begin{equation}
  S_{\text{NY}}[\varepsilon]
  \;=\;
  \int dt\; L_{\text{NY}}(\varepsilon)
  \;\simeq\;
  \int dt\; \Lambda_q\,\varepsilon(t)
  + \mathcal{O}(\varepsilon^3),
  \label{eq:S-NY-eps}
\end{equation}
where the coefficient $\Lambda_q$ is a constant (for a fixed handle)
given schematically by
\begin{equation}
  \Lambda_q
  \;=\;
  -\,\theta_{\text{NY}}\,
  C_{\text{NY}}\,
  q\, m\, \omega\, \Delta r^2,
  \label{eq:Lambda-q-def}
\end{equation}
with $C_{\text{NY}}>0$ encoding the detailed tetrad normalisation and
boundary geometry.  The minus sign in \eqref{eq:Lambda-q-def} matches
the sign convention adopted in \eqref{eq:Lambdaq-sign}.  The precise
derivation of \eqref{eq:Lambda-q-def} and the identification of the positive constant
$C_{\text{NY}}$ are deferred to Appendix~D; for the discussion in Phase~2 we
only need the following qualitative properties:
\begin{itemize}
  \item $\Lambda_q$ is \emph{linear} in the monopole charge $q$,
        with $\Lambda_{-q} = -\Lambda_q$;
  \item $\Lambda_q$ vanishes if either $q=0$ or $\theta_{\text{NY}}=0$;
  \item for fixed $m,\omega,\Delta r^2$ and $\theta_{\text{NY}}$,
        the sign of $\Lambda_q$ is opposite to that of $q$.
\end{itemize}

In the static limit, $S_{\text{NY}}$ contributes to the energy as
\begin{equation}
  V_{\text{NY}}(\varepsilon; q)
  \;=\;
  \Lambda_q\,\varepsilon
  + \mathcal{O}(\varepsilon^3),
  \label{eq:V-NY-eps}
\end{equation}
which is \emph{odd} in $\varepsilon$ and changes sign under
$q \to -q$.

Physically, \eqref{eq:V-NY-eps} plays the rôle of an ``external field''
for the precession mode: it lifts the degeneracy between $\varepsilon>0$
and $\varepsilon<0$ that is present in the TEGR sector.

%------------------------------------------------------------
\subsection{Effective potential and chirality selection}
\label{subsec:phase2-chirality}

Combining the TEGR and Nieh--Yan contributions
\eqref{eq:V-TEGR-eps} and \eqref{eq:V-NY-eps}, we obtain the effective
static potential for the precession mode,
\begin{equation}
  V_{\text{eff}}(\varepsilon; q)
  \;=\;
  E_{\text{TEGR}}(q)
  + \frac{1}{2}k(q)\,\varepsilon^2
  + \Lambda_q\,\varepsilon
  + \mathcal{O}(\varepsilon^4).
  \label{eq:V-eff}
\end{equation}
Minimising with respect to $\varepsilon$ yields
\begin{equation}
  \frac{\partial V_{\text{eff}}}{\partial \varepsilon}
  \;=\;
  k(q)\,\varepsilon + \Lambda_q
  + \mathcal{O}(\varepsilon^3)
  \;=\; 0,
\end{equation}
so that, in the small--$\varepsilon$ regime,
\begin{equation}
  \varepsilon_*(q)
  \;\simeq\;
  -\,\frac{\Lambda_q}{k(q)},
  \qquad
  |\varepsilon_*(q)| \ll 1.
  \label{eq:epsilon-star}
\end{equation}
Using \eqref{eq:Lambdaq-sign}, this may be written as
\begin{equation}
  \varepsilon_*(q)
  \;\simeq\;
  \frac{\gamma}{k(q)}\,q.
\end{equation}
The corresponding shift of the vacuum energy is
\begin{equation}
  V_{\text{eff}}(\varepsilon_*; q)
  \;\simeq\;
  E_{\text{TEGR}}(q)
  - \frac{\Lambda_q^2}{2k(q)},
\end{equation}
so that the Nieh--Yan term lowers the energy by an amount
$\propto \Lambda_q^2/k(q)$ whenever $\Lambda_q\neq 0$.

From \eqref{eq:Lambdaq-sign} it follows that
\begin{equation}
  \varepsilon_*(-q)
  \;\simeq\;
  -\,\varepsilon_*(q),
\end{equation}
for fixed $m,\omega,\theta_{\text{NY}},\Delta r^2$, because
$\Lambda_{-q} = -\Lambda_q$ and $k(q)$ is positive and depends only on
$|q|$.  Thus the sign of the equilibrium tilt is tied to the sign of
the monopole charge: for one sign of $q$ the handle prefers a slightly
``left--tilted'' configuration, while for the opposite sign it prefers
a ``right--tilted'' one.

It is convenient to interpret the sign of $\varepsilon$ as a geometric
proxy for chirality.  In this language, Phase~2 shows that:
\begin{itemize}
  \item in pure TEGR ($\Lambda_q=0$) the two chiralities $\varepsilon>0$
        and $\varepsilon<0$ are exactly degenerate and the vacuum is
        parity--even;
  \item once the Nieh--Yan term is included with nonzero
        $\theta_{\text{NY}}$, the degeneracy is lifted and the ground
        state develops a small chiral bias
        $\varepsilon_*(q)\propto q$;
  \item the direction of this bias (which chirality is favoured) is
        controlled by the sign of $\theta_{\text{NY}}$ and by the
        signs of $m$ and $\omega$ encoded in $\Lambda_q$.
\end{itemize}

In particular, for $q=0$ or $\theta_{\text{NY}}=0$ one has
$\Lambda_q = 0$ and the effective potential reduces to a purely even
function of $\varepsilon$,
\begin{equation}
  V_{\text{eff}}(\varepsilon; q=0)
  \;\simeq\;
  E_{\text{TEGR}}(0)
  + \frac{1}{2}k(0)\,\varepsilon^2
  + \mathcal{O}(\varepsilon^4),
\end{equation}
so that the ground state returns to $\varepsilon_*=0$ and parity is
preserved.  Nonzero $q$ and nonzero $\theta_{\text{NY}}$ are therefore
both necessary ingredients for dynamical chirality selection in this
framework.

The observed unique chirality of weak interactions would, in this
picture, require a cosmological mechanism that selects one sign of
$\theta_{\text{NY}}$ (and hence one global sign of $\Lambda_q$) across
the observable universe.  A concrete realisation of such a mechanism is
left for future work.

%------------------------------------------------------------
\subsection{Range of validity and link to Phase 3}
\label{subsec:phase2-range}

The analysis in this section relies on several approximations:

\begin{itemize}
  \item The precession angle is small,
        $|\varepsilon|\ll 1$, and higher powers
        $\mathcal{O}(\varepsilon^3)$ are neglected.
  \item The precession is spatially uniform along $S^1_\psi$ and $S^2$,
        so that only the zero mode of $\varepsilon(t)$ is retained.
  \item The backreaction of $\varepsilon$ on the radius $r(\psi)$ and on
        the torsion flux $q$ is neglected at leading order.
\end{itemize}

Within this regime, Eq.~\eqref{eq:V-eff} provides a controlled
description of how the Nieh--Yan term biases the precession mode and
induces a small chiral tilt proportional to $q$.  The TEGR part of the
potential is encoded in the stiffness $k(q)$, whose detailed $q$--dependence
is not yet fixed in this section.

In Phase~3 we will analyse $k(q)$ more systematically within the same
Spinning $SU(2)$ handle ansatz, and show that its scaling with $|q|$
leads to a near--critical balance between the TEGR energy $\propto |q|$
and the Nieh--Yan induced energy gain.  This sets the stage for
discussing multi--handle configurations and possible composite states.

\section{Phase 3: Stiffness scaling and critical binding}
\label{sec:phase3}

In Phase~1 (Sec.~\ref{sec:phase1}) we showed that, within the Spinning
$SU(2)$ handle ansatz, pure TEGR provides a parity--even geometric
background with an equal--radius solution whose energy scales as
\begin{equation}
  E_{\text{TEGR}}(q) \;\simeq\; \alpha\,|q|
  \qquad (m\neq 0 \;\text{fixed}),
  \label{eq:E-TEGR-phase3}
\end{equation}
for some positive constant $\alpha$ depending on the twist $m$ and the
handle length $L_\psi$, see Eq.~\eqref{eq:E-TEGR-linear}.  In Phase~2
(Sec.~\ref{sec:phase2}) we showed that the Nieh--Yan term, together with
a slow precession mode $\varepsilon(t)$, induces a chiral tilt
$\varepsilon_*(q) \propto q / k(q)$ and lowers the energy by an amount
\begin{equation}
  \Delta E_{\text{NY}}(q)
  \;\simeq\;
  -\,\frac{\Lambda_q^2}{2k(q)},
  \label{eq:DeltaE-NY-phase3}
\end{equation}
where $\Lambda_q \propto q$ and $k(q)$ is the TEGR--induced stiffness of
the precession mode.

The purpose of Phase~3 is twofold:
\begin{enumerate}
  \item to derive the $q$--dependence of $k(q)$ within the Spinning
        $SU(2)$ handle ansatz and show that, classically,
        \begin{equation}
          k(q)
          \;\propto\;
          \omega^2 m^2 r_0^2
          \;\propto\;
          \omega^2 |q|,
          \label{eq:kq-scaling-intro}
        \end{equation}
        for fixed non--zero twist number $m$, so that the stiffness
        scales linearly with $|q|$;
  \item to analyse the resulting total energy
        \[
          E(q) \;\simeq\; E_{\text{TEGR}}(q) + \Delta E_{\text{NY}}(q)
        \]
        and discuss the near--critical balance between fusion and
        fission of multi--handle configurations.
\end{enumerate}
We will see that the Spinning $SU(2)$ handle sits at a classical
\emph{critical point} where the TEGR energy and the Nieh--Yan induced
binding energy scale with the same power of $|q|$.  Small corrections
to this picture (for example from quantum effects or additional modes)
are then sufficient to tilt the balance slightly and render composite
states with $|q|>1$ energetically competitive with separated unit--charge
handles.

%------------------------------------------------------------
\subsection{Energy components and a scaling ansatz}
\label{subsec:phase3-scaling-ansatz}

From Phase~1 we take the TEGR contribution
\begin{equation}
  E_{\text{TEGR}}(q) \;=\; \alpha\,|q|,
  \qquad \alpha>0,
\end{equation}
for fixed twist $m\neq 0$ and handle length $L_\psi$.

From Phase~2 we take the Nieh--Yan energy gain
\begin{equation}
  \Delta E_{\text{NY}}(q)
  \;=\;
  -\,\frac{\Lambda_q^2}{2k(q)},
\end{equation}
where the precession minimum $\varepsilon_*(q)$ has already been
eliminated via Eq.~\eqref{eq:epsilon-star}.  The coefficient
$\Lambda_q$ is linear in $q$,
\begin{equation}
  \Lambda_q \;=\; -\,\gamma_{\text{NY}}\,q,
  \qquad \gamma_{\text{NY}}>0,
\end{equation}
for fixed $m,\omega,\theta_{\text{NY}},\Delta r^2$,
cf.~Eq.~\eqref{eq:Lambdaq-sign}.  Thus
\begin{equation}
  \Delta E_{\text{NY}}(q)
  \;\propto\;
  -\,\frac{q^2}{k(q)}.
  \label{eq:DeltaE-NY-q2-over-k}
\end{equation}

For the purpose of discussing composite states, it is convenient to
parameterise the stiffness as a power of $|q|$,
\begin{equation}
  k(q)
  \;\propto\;
  |q|^\gamma,
  \label{eq:kq-gamma-ansatz}
\end{equation}
with some exponent $\gamma>0$.  Then
\begin{equation}
  \Delta E_{\text{NY}}(q)
  \;\propto\;
  -\,|q|^{2-\gamma}.
\end{equation}
Up to overall positive constants $\alpha,\beta$, the total energy of a
single handle with charge $q$ can be written as
\begin{equation}
  E(q)
  \;\simeq\;
  \alpha\,|q|
  - \beta\,|q|^{2-\gamma},
  \qquad \alpha,\beta>0,
  \label{eq:E-q-scaling}
\end{equation}
in the regime where the precession mode is well described by the
harmonic approximation and higher--order corrections in
$\varepsilon_*$ can be neglected.

In the remainder of this section we first derive the classical value
$\gamma=1$ within the Spinning $SU(2)$ handle ansatz, and then discuss
its implications for multi--handle binding.

%------------------------------------------------------------
\subsection{Classical derivation of $k(q)\propto \omega^2 |q|$}
\label{subsec:phase3-kq-derivation}

We now sketch how the stiffness scaling
$k(q)\propto \omega^2 |q|$ arises from the TEGR action evaluated on the
precessing ansatz.  The detailed calculations, including the full
$t,\psi$--dependent tetrad and the SymPy implementation, are relegated
to Appendix~\ref{app:phase3-details}; here we focus on the scaling with
$q$, $m$ and $\omega$.

\subsubsection*{(i) Ingredients from Phase~1}

From Phase~1 we recall that for the equal--radius background, with
Spinning $SU(2)$ handle ansatz and twist number $m$, the effective
TEGR energy reduces to
\begin{equation}
  E_{\text{TEGR}}(q)
  \;=\;
  L_\psi\,V(r_0),
\end{equation}
with
\begin{equation}
  V(r)
  \;=\;
  \frac{B q^2}{r^2} + C m^2 r^2,
\end{equation}
and the equilibrium radius
\begin{equation}
  r_0
  \;=\;
  \left(\frac{B}{C}\right)^{1/4}
  \frac{|q|^{1/2}}{|m|^{1/2}}.
  \label{eq:r0-phase3}
\end{equation}
For fixed non--zero $m$ we may absorb the $m$--dependence into the
numerical prefactor and write, at the level of scaling,
\begin{equation}
  r_0^2 \;\propto\; |q|.
  \label{eq:r0sq-propto-q}
\end{equation}

\subsubsection*{(ii) Precession as a homogeneous rotation}

In Phase~2 we implemented the precession mode as an additional
time--dependent $SU(2)$ rotation,
\begin{equation}
  U_{\text{prec}}(t)
  \;=\;
  \exp\!\left(\frac{i}{2}\,\varepsilon(t)\,\sigma_2\right),
\end{equation}
acting on the left of the background configuration, see
Eq.~\eqref{eq:U-with-prec}.  Via the $SU(2)\to SO(3)$ map, this induces
a rotation of the spatial triad about the internal 2--axis by the angle
$\varepsilon(t)$, tilting the handle axis away from the $\psi$--direction
by $\varepsilon(t)$ while preserving the equal--radius condition
$r(\psi)=r_0$.

To quadratic order in $\varepsilon$ and its time derivative, the TEGR
action evaluated on this precessing tetrad contains terms of the form
\begin{equation}
  S_{\text{TEGR}}
  \;\supset\;
  \int dt\,
  \Bigl[
    \frac{1}{2}I(q)\,\dot{\varepsilon}^2
    - \frac{1}{2}k(q)\,\varepsilon^2
  \Bigr],
\end{equation}
where $I(q)$ is an effective ``moment of inertia'' for the precession
mode and $k(q)$ is the stiffness appearing in
Eq.~\eqref{eq:V-TEGR-eps}.  Both $I(q)$ and $k(q)$ are integrals over
$S^1_\psi\times S^2$ of quadratic combinations of the torsion
components induced by the precession.

\subsubsection*{(iii) Scaling of the stiffness with $q$, $m$ and $\omega$}

A direct expansion of the torsion scalar $\mathbb{T}$ in powers of
$\varepsilon$ shows that the dominant contribution to $k(q)$ takes the
schematic form
\begin{equation}
  k(q)
  \;\sim\;
  C_k\,
  \omega^2 m^2\,r_0^2,
  \qquad C_k>0,
  \label{eq:kq-m2r0sq}
\end{equation}
where $C_k$ is a numerical coefficient depending on the precise
normalisation and on the integration over $\psi$ and $S^2$.  Intuitively,
$k(q)$ measures how much TEGR energy is stored in bending a spinning,
twisted handle away from the $\psi$--axis: the factor $m^2$ reflects the
cost of tilting a strongly twisted configuration, the factor $r_0^2$
provides the lever arm, and the factor $\omega^2$ reflects the fact that
precession couples to the underlying spin.

Using the Phase~1 scaling \eqref{eq:r0sq-propto-q}, we obtain
\begin{equation}
  k(q)
  \;\propto\;
  \omega^2 m^2\,r_0^2
  \;\propto\;
  \omega^2 m^2\,|q|.
\end{equation}
For fixed non--zero $m$ this implies
\begin{equation}
  k(q)
  \;\propto\;
  \omega^2 |q|,
\end{equation}
so that the exponent in the scaling ansatz
\eqref{eq:kq-gamma-ansatz} is
\begin{equation}
  \gamma_{\text{classical}} = 1.
  \label{eq:gamma-equals-one}
\end{equation}

This is the central result of Phase~3 at the classical level: within the
Spinning $SU(2)$ handle ansatz, the stiffness of the precession mode is
linear in $|q|$, with a prefactor controlled by $\omega^2$ (and by $m^2$
through the overall constant).

%------------------------------------------------------------
\subsection{Critical binding and multi--handle configurations}
\label{subsec:phase3-critical-binding}

Combining Eqs.~\eqref{eq:E-TEGR-phase3},
\eqref{eq:DeltaE-NY-q2-over-k}, and \eqref{eq:kq-m2r0sq}, the total energy
of a single handle with charge $q$ can be written as
\begin{equation}
  E(q)
  \;\simeq\;
  \alpha\,|q|
  - \beta\,|q|^{2-\gamma},
  \qquad
  \gamma = 1,
  \label{eq:E-q-gamma1}
\end{equation}
for suitable positive constants $\alpha,\beta$ that encode the details
of the TEGR and Nieh--Yan couplings as well as $m$ and $\omega$.
Explicitly, for $\gamma=1$ this is
\begin{equation}
  E(q)
  \;\simeq\;
  (\alpha-\beta)\,|q|.
  \label{eq:E-q-linear-effective}
\end{equation}
Equivalently, the energy \emph{per unit charge} is
\begin{equation}
  \frac{E(q)}{|q|}
  \;\simeq\;
  \alpha - \beta,
\end{equation}
which is independent of $q$ at leading order.  This ``flatness'' of
$E(q)/|q|$ in $q$ is the hallmark of a critical point: neither strong
binding ($\gamma<1$) nor strong repulsion ($\gamma>1$) is favoured
classically, so that even tiny subleading corrections can decide whether
states with $|q|>1$ are slightly bound or slightly unbound.

To discuss composite states we compare:
\begin{itemize}
  \item a single handle with total charge $q=N$,
  \item $N$ well--separated handles, each with unit charge $q=1$.
\end{itemize}
Using \eqref{eq:E-q-scaling}, the energy of a single $|q|=N$ handle is
\begin{equation}
  E(N)
  \;=\;
  \alpha N - \beta N^{2-\gamma},
\end{equation}
while that of $N$ separated $|q|=1$ handles is
\begin{equation}
  N E(1)
  \;=\;
  N(\alpha - \beta).
\end{equation}
The binding energy of the composite state is therefore
\begin{equation}
  E_{\text{bind}}(N)
  \;:=\;
  E(N) - N E(1)
  \;=\;
  \beta N \left(1 - N^{1-\gamma}\right).
  \label{eq:Ebind-general}
\end{equation}
For $N>1$ and $\beta>0$ one finds:
\begin{itemize}
  \item If $\gamma<1$, then $1-\gamma>0$ and $N^{1-\gamma}>1$, so
        $E_{\text{bind}}(N)<0$: composite states with any $N>1$ are
        energetically favoured.
  \item If $\gamma>1$, then $1-\gamma<0$ and $N^{1-\gamma}<1$, so
        $E_{\text{bind}}(N)>0$: all composite states are unfavoured.
  \item If $\gamma=1$, then $N^{1-\gamma}=1$ and
        $E_{\text{bind}}(N)=0$: all composite states are exactly
        marginal at this order.
\end{itemize}

Thus the value $\gamma=1$ derived in
Eq.~\eqref{eq:gamma-equals-one} corresponds to a
\emph{critical point} separating a regime where fusion of handles is
generically favoured ($\gamma<1$) from a regime where it is generically
disfavoured ($\gamma>1$).  At the classical level of the present ansatz,
the Spinning $SU(2)$ handle sits precisely at this critical point.

Physically, this means that:
\begin{itemize}
  \item the TEGR energy, scaling as $\propto |q|$, behaves as a tension
        that penalises large $|q|$;
  \item the Nieh--Yan induced attraction, scaling as
        $\propto - |q|^{2-\gamma}$, competes with this tension;
  \item for $\gamma=1$ the competition is exactly balanced in the
        leading power of $|q|$, and subleading corrections decide
        whether composite states are slightly bound or slightly unbound.
\end{itemize}

%------------------------------------------------------------
\subsection{$q=3$ baryon--like states and the rôle of corrections}
\label{subsec:phase3-q3}

The discussion above shows that within the classical Spinning $SU(2)$
handle ansatz, the exponent $\gamma$ takes the critical value $\gamma=1$.
At this level, Eq.~\eqref{eq:Ebind-general} predicts
\begin{equation}
  E_{\text{bind}}(N) \;\simeq\; 0
  \qquad (N>1),
\end{equation}
so that composite states are neither clearly favoured nor clearly
disfavoured by the leading scaling behaviour alone.

In order for a specific composite state, such as $|q|=3$, to become
\emph{energetically preferred} over three isolated $|q|=1$ handles, we
require small corrections that effectively shift the exponent away from
its classical value,
\begin{equation}
  \gamma_{\text{eff}} \;\lesssim\; 1,
\end{equation}
at least in the range of $q$ relevant to low--charge composites.  In
that case,
\begin{equation}
  E_{\text{bind}}(N)
  \;\simeq\;
  \beta N \left(1 - N^{1-\gamma_{\text{eff}}}\right)
  \;<\; 0
  \qquad (N>1),
\end{equation}
and composite states become slightly bound.

Within the present paper we do not compute $\gamma_{\text{eff}}$ beyond
the classical level; however, there are several plausible sources of
such corrections:
\begin{itemize}
  \item quantum fluctuations of the precession mode $\varepsilon(t)$ and
        of the radius $r(\psi)$;
  \item additional geometric modes that have been frozen in the present
        ansatz (for example bending of the handle or inhomogeneous
        deformations);
  \item couplings to matter fields or to a ``colour'' degree of freedom
        that distinguish different arrangements of handles with the same
        total $q$.
\end{itemize}

The main message of Phase~3 is therefore deliberately modest but
encouraging:
\begin{quote}
  Within the Spinning $SU(2)$ handle ansatz, the classical teleparallel
  stiffness sits exactly at the critical scaling point $\gamma=1$,
  where TEGR tension and Nieh--Yan attraction compete on equal footing.
  Small corrections --- for example from quantum effects --- are then
  sufficient, in principle, to render composite states with $|q|\ge 2$
  (in particular $|q|=3$) energetically competitive with separated
  unit--charge handles.
\end{quote}
A detailed exploration of such corrections, and a possible geometric
selection of $|q|=3$ as a particularly stable ``baryon--like'' charge,
is left for future work.

%============================================================
\section{Discussion}
\label{sec:discussion}

In this section we place the results of Phases~1--3 in a broader
context, emphasising both the suggestive analogies with known particle
physics and the limitations of the present toy model.

%------------------------------------------------------------
\subsection{Relation to Skyrmions and QCD--like phenomenology}
\label{subsec:discussion-skyrmions}

The microscopic handle picture explored in this paper is reminiscent of
Skyrmion models of baryons in several respects:

\begin{itemize}
  \item In the Skyrme model, baryon number arises as a topological
        charge of an $SU(2)$--valued field on spatial $S^3$, and the
        Wess--Zumino--Witten (WZW) term controls the statistics of
        Skyrmions.  In the present work, the monopole charge $q$ plays
        an analogous rôle as a topological charge associated with torsion
        flux, while the Nieh--Yan term plays a rôle similar to that of a
        WZW term in inducing parity--odd effects.
  \item The stability of Skyrmions is governed by a competition between
        gradient/tension terms and topological terms in the effective
        action.  Here, the TEGR energy $\propto |q|$ and the
        Nieh--Yan--induced energy gain
        $\Delta E_{\text{NY}}\propto -q^2/k(q)$ compete in an analogous
        way, with the stiffness $k(q)$ playing the rôle of a geometric
        coupling.
  \item In Skyrme phenomenology, $|B|=1$ and $|B|=3$ configurations are
        of special interest as nucleons and baryons.  In our setting,
        single--handle states with $|q|=1$ and composite states with
        $|q|=3$ are natural analogues of ``quark--like'' and
        ``baryon--like'' structures.
\end{itemize}

At the same time, there are important differences and limitations:

\begin{itemize}
  \item The present model is formulated purely in terms of geometry and
        torsion in teleparallel gravity.  Standard Model fields
        (quarks, leptons, gauge bosons) are not dynamically included;
        any connection to real QCD or electroweak physics is therefore
        indirect and speculative at this stage.
  \item Our analysis focuses on a highly symmetric ansatz (Spinning
        $SU(2)$ handle) and a small number of collective coordinates.
        Generic deformations, interactions between multiple handles, and
        the full spectrum of excitations have not been explored.
  \item We have worked entirely at the classical level in the gravity
        sector.  Quantum corrections, renormalisation of couplings, and
        the embedding into a UV--complete theory remain open issues.
\end{itemize}

Fermionic statistics of odd--$q$ states itself is expected to arise from
an odd WZW level induced by the Nieh--Yan charge on the handle boundary,
in the spirit of Chandia--Zanelli and Witten
(see Sec.~\ref{sec:TEGR} and Refs.~\cite{Chandia:1997hu,Witten:1983tw}).

We therefore view the handle picture not as a realistic model of
baryons, but as a geometric toy model that reproduces a subset of the
qualitative features of Skyrmion physics in a purely teleparallel
setting.  The analogies are intriguing, but any phenomenological
interpretation must be made with caution.

%------------------------------------------------------------
\subsection{Limitations of the present ansatz and possible extensions}
\label{subsec:discussion-limitations}

The Spinning $SU(2)$ handle ansatz has been chosen for its mathematical
simplicity and its ability to make the Nieh--Yan structure explicit.
However, this simplicity comes with limitations:

\begin{itemize}
  \item \textbf{Restricted degrees of freedom.}  We have fixed the
        handle to be straight, with constant radius $r_0$, and allowed
        only a global precession mode.  Bending modes, inhomogeneous
        perturbations along $S^1_\psi$, and more general shape
        deformations could change the energy landscape and the stiffness
        scaling.
  \item \textbf{Single--handle focus.}  Our analysis of composite states
        has treated multi--handle configurations in a coarse--grained
        way, essentially through a scaling argument in $q$.  A more
        realistic treatment would require explicit multi--handle
        solutions and an analysis of their interactions.
  \item \textbf{Neglect of backreaction.}  We have assumed that the
        backreaction of the precession mode on the radius $r(\psi)$ and
        on the surrounding bulk geometry is small.  For large $|q|$ or
        in dense handle configurations this assumption may break down.
\end{itemize}

There are several natural directions in which the present ansatz could
be extended:

\begin{itemize}
  \item Allowing the radius $r(\psi)$ to vary along the handle and
        including bending modes, leading to a richer spectrum of
        collective coordinates.
  \item Coupling the handle geometry to matter fields, in particular
        spinor fields, to explore how fermionic degrees of freedom
        propagate on or are localised by the handle.
  \item Embedding the microscopic handle picture into a cosmological
        background, where a finite density of handles could backreact on
        the large--scale evolution of the universe.
\end{itemize}

We leave these extensions to future work; they are mentioned here to
clarify the scope of the present paper.

%------------------------------------------------------------
\subsection{Future directions: meson--like states, quantum corrections, and cosmology}
\label{subsec:discussion-future}

Finally, we briefly outline some concrete directions for future
investigation suggested by the present analysis:

\begin{enumerate}
  \item \textbf{Meson--like states.}  Composite configurations with
        total charge $q_{\text{tot}}=0$, built for example from a
        $q=+1$ and a $q=-1$ handle, are natural candidates for
        ``meson--like'' states in this geometric picture.  Their
        stability and spectrum depend sensitively on the interaction
        between handles of opposite charge and on the sign of the
        Nieh--Yan coupling.
  \item \textbf{Quantum corrections and $\gamma_{\text{eff}}$.}  A more
        systematic computation of quantum corrections to the stiffness
        $k(q)$ and to the TEGR and Nieh--Yan couplings could determine
        whether $\gamma_{\text{eff}}$ is indeed slightly below 1 in a
        realistic setting, and whether this is sufficient to render
        $|q|=3$ and other composites energetically competitive.
  \item \textbf{Spinor sector and effective field theory.}  By
        integrating out microscopic degrees of freedom on a network of
        handles, one might arrive at an effective field theory in which
        the handle charge plays the rôle of a topological quantum
        number, with an emergent chiral fermion spectrum.  This would
        bring the handle picture closer to phenomenological models.
  \item \textbf{Cosmological implications.}  If a finite density of
        microscopic handles is present in the early universe, their
        collective dynamics and phase transitions (for example across
        the critical point $\gamma=1$) could leave imprints on
        cosmological observables.  Exploring such scenarios would
        require coupling the present model to a dynamical FRW background.
\end{enumerate}

These directions go well beyond the scope of the present paper, but they
illustrate how the simple geometric mechanism studied here could serve
as a starting point for more ambitious constructions.

%============================================================
\section{Conclusions}
\label{sec:conclusions}

In this paper we have explored a geometric toy model in which microscopic
wormhole--like handles in teleparallel gravity, described by a Spinning
$SU(2)$ ansatz, give rise to a rich interplay between topology,
chirality, and binding.

Our main results can be summarised as follows:
\begin{itemize}
  \item \textbf{Phase~1 (Sec.~\ref{sec:phase1}):}  Within pure TEGR, the
        Spinning $SU(2)$ handle ansatz admits a classically stable
        equal--radius configuration with radius
        $r_0 \propto |q|^{1/2}$ (for fixed twist $m$) and TEGR energy
        $E_{\text{TEGR}}(q)\propto |q|$.  This provides a parity--even
        geometric background in which the sign of $q$ is invisible at
        the level of the TEGR sector alone.
  \item \textbf{Phase~2 (Sec.~\ref{sec:phase2}):}  Turning on the
        Nieh--Yan term and introducing a slow precession mode
        $\varepsilon(t)$ leads to an effective potential
        \[
          V_{\text{eff}}(\varepsilon; q)
          =
          E_{\text{TEGR}}(q)
          + \frac{1}{2}k(q)\,\varepsilon^2
          + \Lambda_q\,\varepsilon
          + \cdots,
        \]
        where $k(q)>0$ is the TEGR stiffness and $\Lambda_q\propto q$
        encodes the Nieh--Yan coupling.  The ground state develops a
        small chiral tilt $\varepsilon_*(q)\propto q/k(q)$, so that the
        sign of the precession is tied to the sign of the topological
        charge $q$.
  \item \textbf{Phase~3 (Sec.~\ref{sec:phase3}):}  Evaluating the TEGR
        action on the precessing ansatz shows that the stiffness scales
        as $k(q)\propto \omega^2 |q|$ for fixed twist $m$, i.e.\ the
        exponent in $k(q)\propto |q|^\gamma$ is $\gamma=1$ at the
        classical level.  As a consequence, the TEGR tension and the
        Nieh--Yan induced attraction scale with the same power of
        $|q|$, placing the model at a critical point between generic
        fusion and generic fission of handles.  Small corrections (for
        example quantum effects) can then, in principle, render
        low--charge composites with $|q|\ge 2$ energetically competitive
        with separated unit--charge handles.
\end{itemize}

These results do \emph{not} yet constitute a realistic theory of
fermions or of the Standard Model.  Rather, they demonstrate that within
a simple teleparallel setting, a single geometric mechanism --- the
interaction between TEGR and the Nieh--Yan term on a microscopic handle
--- can simultaneously:
\begin{itemize}
  \item endow topological defects with a parity--even ``mass''
        proportional to $|q|$;
  \item generate a parity--odd tilt whose sign is locked to the sign of
        $q$;
  \item and place the system near a critical point for the binding of
        composite configurations.
\end{itemize}

We hope that this toy model can serve as a useful stepping stone toward
more complete frameworks in which fermionic statistics, chirality, and
binding emerge from the geometry and topology of spacetime itself.
Further work will be required to incorporate matter fields, quantum
effects, and cosmological dynamics, and to assess whether the hints
found here can be developed into quantitatively predictive models.


%-----------------------------
% Code availability
%-----------------------------

\section*{Code availability}
All symbolic computations in this work were performed using custom
SymPy scripts. These scripts are included as supplementary files in the
Zenodo record associated with this preprint:
\href{https://doi.org/10.5281/zenodo.17762205}{doi:10.5281/zenodo.17762205}.


%-----------------------------
% Acknowledgments
%-----------------------------

\section*{Acknowledgments}

The author is deeply grateful to the AI assistants Grok (xAI), ChatGPT (OpenAI),
Gemini (Google), and Claude (Anthropic) for their invaluable assistance with
symbolic computations, consistency checks, geometric visualizations, and
critical scientific feedback throughout this work.

This research was carried out under the informal collaborative project
\textbf{DPPU} --- standing for
\textbf{D}onut-like topology ($S^2 \times S^1$),
\textbf{P}lanck-scale compactness,
\textbf{P}recession dynamics,
within the context of our observable
\textbf{U}niverse.

The name simultaneously reflects the microscopic handle geometry that emerged
somewhat serendipitously in early discussions, and the joyful, almost trembling
excitement that accompanied each small breakthrough.

All remaining errors are, of course, the author's responsibility.

%-----------------------------
% Appendices
%-----------------------------

\appendix

%%%%%%%%%%%%%%%%%%%%%%%%%%%%%%%%%%%%%%%%%%%%%%%%%%%%%%%%%%%%%%%%%%%%%%%%%%%%%%
\section{Notation, Conventions, and Dimensional Analysis}
\label{app:notation}
%%%%%%%%%%%%%%%%%%%%%%%%%%%%%%%%%%%%%%%%%%%%%%%%%%%%%%%%%%%%%%%%%%%%%%%%%%%%%%

We work in natural Planck units
\begin{equation}
\hbar = c = 8\pi G = 1,
\end{equation}
so that length, time and inverse mass all have the same dimension,
\([L]=[T]=[M]^{-1}\).  The Planck length is denoted by
$\ell_{\rm Pl}$.

Throughout the paper we use the mostly–plus signature
$(-,+,+,+)$ and Latin indices $a,b,\dots$ for tangent–space indices,
Greek indices $\mu,\nu,\dots$ for spacetime indices.
The metric is obtained from the tetrad by
\begin{equation}
g_{\mu\nu} = e^a{}_\mu e^b{}_\nu \eta_{ab},
\qquad
\eta_{ab} = \mathrm{diag}(-1,1,1,1).
\end{equation}

\subsection{Physical parameters and their dimensions}

The basic parameters appearing in our handle ansatz are summarised
in Table~\ref{tab:dimensions}.  When needed we explicitly restore
powers of $\ell_{\rm Pl}$; otherwise we simply treat all dimensionful
quantities as measured in Planck units.

\begin{table}[h]
  \centering
  \begin{tabular}{lll}
    \hline
    Symbol & Meaning & Dimension \\
    \hline
    $q$         & torsional monopole charge (Nieh--Yan charge) & dimensionless ($\mathbb{Z}$) \\
    $m$         & twist winding number                          & dimensionless ($\mathbb{Z}$) \\
    $\omega$    & background spin frequency                     & $[T]^{-1}\sim [\ell_{\rm Pl}]^{-1}$ \\
    $r_0$       & handle radius                                 & $[L]\sim[\ell_{\rm Pl}]$ \\
    $L_\psi$    & length of compact direction $\psi$            & $[L]$ \\
    $\varepsilon$ & small precession angle                      & dimensionless \\
    $\theta_{\rm NY}$ & Nieh--Yan coupling                      & $[\ell_{\rm Pl}]^{-2}$ \\
    \hline
  \end{tabular}
  \caption{Physical parameters and their dimensions.}
  \label{tab:dimensions}
\end{table}

The TEGR action is
\begin{equation}
S_{\rm TEGR}
 = \frac{1}{2\kappa}\int d^4x\, e\,\mathbb{T},
\qquad
\kappa = 8\pi G = 1,
\end{equation}
where $e=\det(e^a{}_\mu)$ and $\mathbb{T}$ is the torsion scalar.
Since $S_{\rm TEGR}$ is dimensionless, we have
\([e\,d^4x]\sim[\ell_{\rm Pl}]^4\) and $[\mathbb{T}]\sim[\ell_{\rm Pl}]^{-2}$.
For a configuration of characteristic size $r_0$ this gives the
parametric estimate
\begin{equation}
E_{\rm TEGR}
\sim \ell_{\rm Pl}^{-1}\times (\text{dimensionless function of }q,m,\omega),
\end{equation}
consistent with the energy per handle scaling as $E\propto |q|$ in the
main text.

For the precession mode $\varepsilon(t)$ we eventually obtain an effective
action of the form
\begin{equation}
S_{\rm eff}^{(\varepsilon)}
= \frac{1}{2}\int dt\, k(q)\,\varepsilon(t)^2 + \cdots ,
\end{equation}
so $k(q)$ has dimension $[\ell_{\rm Pl}]^{-1}$.
In our ansatz this stiffness scales as
\begin{equation}
k(q)
\sim m^2\omega^2 r_0^2 L_\psi
\sim \omega^2 |q|,
\end{equation}
where we used $r_0^2\propto |q|$ (with $m$ fixed).

The Nieh--Yan term is
\begin{equation}
S_{\rm NY}
  = \theta_{\rm NY}\int_M \mathcal{N},
\end{equation}
with $\mathcal{N}$ the Nieh--Yan density.  For a single handle the
integral scales as $\int_M\mathcal{N}\sim q\,\Delta r^2\sim
q\,\ell_{\rm Pl}^2$, so that $[\theta_{\rm NY}]=[\ell_{\rm Pl}]^{-2}$,
in agreement with Table~\ref{tab:dimensions}.

All coefficients $A,B,C,\dots$ that appear in the 1D effective
description in Phase~1 are understood as $\mathcal{O}(1)$ numerical
constants in these Planck units.



%%%%%%%%%%%%%%%%%%%%%%%%%%%%%%%%%%%%%%%%%%%%%%%%%%%%%%%%%%%%%%%%%%%%%%%%%%%%%%
\section{Explicit Tetrad: Background Spin/Twist and Precession}
\label{app:tetrad}
%%%%%%%%%%%%%%%%%%%%%%%%%%%%%%%%%%%%%%%%%%%%%%%%%%%%%%%%%%%%%%%%%%%%%%%%%%%%%%

\subsection{Static monopole reference tetrad}

On $\mathbb{R}_t\times S^1_\psi\times S^2$ we use coordinates
$(t,\psi,\theta,\phi)$.
A convenient static reference tetrad carrying monopole charge $q$
is
\begin{align}
\tilde e^0 &= dt, \label{eq:ref-static-0}\\
\tilde e^1 &= d\psi, \label{eq:ref-static-1}\\
\tilde e^2 &= r_0\, d\theta, \label{eq:ref-static-2}\\
\tilde e^3 &= r_0\sin\theta\,\bigl[d\phi + q(1-\cos\theta)\,d\psi\bigr].
\label{eq:ref-static-3}
\end{align}
This tetrad reproduces the usual monopole gauge potential on $S^2$
in the $(\theta,\phi)$ directions and an $S^1$ fibre parametrised by
$\psi$.

\subsection{Background spinning--twisted tetrad}

The physical background used in the main text combines
\emph{spin} around the handle axis and a \emph{twist} along
$S^1_\psi$.
At the level of the internal $SU(2)$ frame this can be represented
schematically by
\begin{equation}
U(t,\psi)=
\exp\!\left( \frac{i}{2}\omega t\,\sigma_3\right)
\exp\!\left( \frac{i}{2}m\psi\,\sigma_3\right),
\end{equation}
with $\sigma_3$ a Pauli matrix.
The background tetrad $e^a{}_\mu{}_{\rm bg}$ is obtained by acting
with the corresponding $SO(3)$ rotation on the spatial triad
$\tilde e^{i}$ ($i=1,2,3$).  The explicit expressions are not needed
here; the important point is that they preserve axial symmetry and
encode the monopole charge $q$, twist $m$ and spin frequency $\omega$.

\subsection{Precession as a small axis tilt}

To model precession we introduce a slow, spatially uniform tilt
of the handle axis.  In a frame adapted to the background, the
tilt can be implemented by an infinitesimal $SO(3)$ rotation that
mixes the $e^1$ and $e^3$ directions:
\begin{equation}
e^a = \Lambda^a{}_b(\varepsilon(t))\,e^b_{\rm bg},
\qquad
\Lambda(\varepsilon) =
\begin{pmatrix}
1 & 0 & 0 & 0 \\
0 & \cos\varepsilon & 0 & \sin\varepsilon \\
0 & 0 & 1 & 0 \\
0 & -\sin\varepsilon & 0 & \cos\varepsilon
\end{pmatrix},
\end{equation}
with $\varepsilon(t)\ll 1$.
To first order in $\varepsilon$ one has
\begin{align}
e^1 &\simeq e^1_{\rm bg} + \varepsilon(t)\, e^3_{\rm bg},\\
e^3 &\simeq e^3_{\rm bg} - \varepsilon(t)\, e^1_{\rm bg},
\end{align}
while $e^0$ and $e^2$ are unchanged.
Thus the handle axis, originally aligned with $d\psi$, acquires a
small time–dependent tilt of order $\varepsilon(t)$.

In the explicit stiffness calculation of
Appendix~\ref{app:phase3-calc} it is technically convenient to
work in a rotated spatial frame in which the same physical precession
is represented as a rotation in the $(e^2,e^3)$–plane while $e^1$
is kept fixed.  This change of frame does not affect any physical
observable, but simplifies the algebra in the SymPy implementation.



%%%%%%%%%%%%%%%%%%%%%%%%%%%%%%%%%%%%%%%%%%%%%%%%%%%%%%%%%%%%%%%%%%%%%%%%%%%%%%
\section{Phase 1: Detailed Derivation of the Effective Radial Potential}
\label{app:phase1}
%%%%%%%%%%%%%%%%%%%%%%%%%%%%%%%%%%%%%%%%%%%%%%%%%%%%%%%%%%%%%%%%%%%%%%%%%%%%%%

\subsection{TEGR action on the spinning--twisted ansatz}

Evaluating the TEGR action on the spinning--twisted background
tetrad of Sec.~\ref{app:tetrad} and imposing the equal–radius ansatz
$r(\psi)=r_0$ yields an effective one–dimensional functional of the
form
\begin{equation}
S_{\rm TEGR}
= \int dt\int_0^{L_\psi} d\psi \,
\left[
 A\,(\partial_\psi r)^2 + V(r)
\right],
\end{equation}
where $A>0$ is a numerical coefficient and $V(r)$ is an effective
radial potential.  For the equal–radius sector we may set
$\partial_\psi r=0$, so that the energy per unit $\psi$–length
reduces to
\begin{equation}
\mathcal{E} = \int_0^{L_\psi} d\psi\, V(r_0)
 = L_\psi\,V(r_0).
\end{equation}

\subsection{Dominant contributions to $V(r)$}

A straightforward but algebraically lengthy evaluation of the torsion
scalar on this ansatz shows that the dominant contributions to
$V(r)$ are
\begin{align}
V_{\rm core}(r) &= B\,\frac{q^2}{r^2},\\
V_{\rm twist}(r) &= C\,m^2 r^2,
\end{align}
with $B,C>0$ numerical coefficients of order unity in the Planck
units defined in Appendix~\ref{app:notation}.  The first term is
supported by the monopole–like torsion generated by $q$, while the
second term comes from the anisotropic ``twist'' gradients set by
$m$ along the fibre direction.

Higher–derivative corrections and subleading powers of $r$ are
suppressed for the nearly uniform configurations of interest and are
therefore dropped at this stage.

\subsection{Minimisation and stable radius}

The total potential is
\begin{equation}
V(r) = V_{\rm core}(r) + V_{\rm twist}(r)
     = B\,\frac{q^2}{r^2} + C\,m^2 r^2.
\end{equation}
Minimising with respect to $r$ gives
\begin{equation}
\frac{dV}{dr} = -2B\,\frac{q^2}{r^3} + 2C m^2 r = 0,
\end{equation}
so that the equilibrium radius satisfies
\begin{equation}
r_0^4 = \frac{B}{C}\,\frac{q^2}{m^2}
\qquad\Rightarrow\qquad
r_0^2 \propto \frac{|q|}{|m|}.
\end{equation}
The second derivative at the extremum,
\begin{equation}
\frac{d^2V}{dr^2}\bigg|_{r=r_0}
 = 6B\,\frac{q^2}{r_0^4} + 2C m^2 > 0,
\end{equation}
is strictly positive, confirming that this configuration is a
classical minimum.

Substituting $r_0$ back into $V(r_0)$ we obtain
\begin{equation}
E_{\rm TEGR}(q)
 = L_\psi V(r_0)
 = \alpha\,|q|,
\qquad
\alpha = 2\sqrt{BC}\,|m|\,L_\psi > 0,
\end{equation}
showing that the TEGR energy is proportional to $|q|$, as stated in
Sec.~4 of the main text.  The equal–radius approximation is
self–consistent as long as the curvature and torsion scales are
super–Planckian, which is the regime of interest in this toy model.



%%%%%%%%%%%%%%%%%%%%%%%%%%%%%%%%%%%%%%%%%%%%%%%%%%%%%%%%%%%%%%%%%%%%%%%%%%%%%%
\section{Nieh--Yan Boundary Term and Precession Coupling}
\label{app:phase2}
%%%%%%%%%%%%%%%%%%%%%%%%%%%%%%%%%%%%%%%%%%%%%%%%%%%%%%%%%%%%%%%%%%%%%%%%%%%%%%

Starting from the explicit spinning--twisted tetrad, one finds that
the Nieh--Yan density on a single handle takes the exact form
\begin{equation}
\mathcal{N}
 = q\,\omega m\,
   d\!\bigl(r(\psi)^2\bigr)\wedge dt\wedge
   \sin\theta\,d\theta\wedge d\phi
 \quad
 (\text{up to a positive numerical factor}),
\end{equation}
where $r(\psi)$ is the local radius of the $S^2$ cross–section and
$\omega$ is the background spin frequency.  Integrating over the
two–sphere and along the handle we obtain
\begin{equation}
\int_M \mathcal{N}
 = 4\pi\,q\,\omega m\,\Delta r^2,
\qquad
\Delta r^2 \equiv
 r^2(\psi_{\rm end}) - r^2(\psi_{\rm start}),
\end{equation}
again up to an overall positive factor that we absorb into the
definition of the coupling $\theta_{\rm NY}$.

For an exactly equal–radius configuration we have $\Delta r^2=0$
and therefore
\begin{equation}
\int_M\mathcal{N}=0
\quad\Rightarrow\quad
S_{\rm NY}=0.
\end{equation}
This confirms the statement in the main text that the spinning
background by itself remains parity even; parity violation arises
only once we take into account boundary effects associated with
precession.

\subsection{Precession--induced boundary mismatch}

When the handle precesses with a small, slowly varying angle
$\varepsilon(t)$, the junction between the microscopic handle and
the external ``bulk'' geometry develops a small mismatch.
To first order in $\varepsilon$ the difference in the effective
$r^2$ between the two ends of the handle can be parameterised as
\begin{equation}
\Delta r^2
\;\longrightarrow\;
\Delta r^2 + \beta\,\varepsilon(t),
\end{equation}
where $\beta>0$ is a model–dependent constant encoding how the
junction geometry responds to a uniform tilt.\footnote{A concrete
realisation of this behaviour can be given in a multi--handle
geometry where the external metric is kept fixed while the
microscopic handle is rotated.  The precise value of $\beta$ is not
needed in the main text; only its sign matters.}
Substituting into the Nieh--Yan term we obtain a linear contribution
to the effective action for $\varepsilon(t)$,
\begin{equation}
S_{\rm NY}^{(1)}
 = \theta_{\rm NY}\int dt\,
 \Lambda_q\,\varepsilon(t),
\qquad
\Lambda_q
 = {\sf c}_{\rm NY}\,q,
\end{equation}
with
${\sf c}_{\rm NY} \sim \theta_{\rm NY}\,\omega m\beta$.
We choose the sign convention such that
\begin{equation}
\Lambda_q = -\gamma\,q,
\qquad \gamma>0,
\end{equation}
so that the effective potential
$V_{\rm eff}(\varepsilon;q)$ is minimised at a small tilt
$\varepsilon_*(q)$ whose sign is locked to the sign of $q$,
as discussed in Sec.~5.

\begin{figure}[t]
  \centering
  \includegraphics[width=0.9\linewidth]{figs/Fig1_Junction_Mismatch.png}
  \caption{Schematic representation of the Nieh--Yan boundary
  mechanism.  Panel (a): aligned handle ($\varepsilon=0$) with
  identical junctions at both ends, $\Delta r^2=0$ and no Nieh--Yan
  contribution.  Panel (b): precessing handle ($\varepsilon\neq 0$);
  the junction at the upper end no longer matches the external
  geometry (red contour), producing a small shift
  $\Delta r^2\to\Delta r^2+\beta\varepsilon$ and hence a linear term
  $\Lambda_q\varepsilon$ in the effective action via the Nieh--Yan
  density.}
  \label{fig:junction}
\end{figure}

The key point is that the entire parity–violating effect is of
\emph{purely boundary origin}.  No bulk contribution to the Nieh--Yan
density is required once the equal–radius condition is imposed.



%%%%%%%%%%%%%%%%%%%%%%%%%%%%%%%%%%%%%%%%%%%%%%%%%%%%%%%%%%%%%%%%%%%%%%%%%%%%%%
\section{Complete Calculation of the Precession Stiffness $k(q)$}
\label{app:phase3-calc}
%%%%%%%%%%%%%%%%%%%%%%%%%%%%%%%%%%%%%%%%%%%%%%%%%%%%%%%%%%%%%%%%%%%%%%%%%%%%%%

\subsection{Tetrad with precession in a convenient frame}

For the explicit computation of the stiffness $k(q)$ we adopt a
frame in which the precession is realised as a rotation in the
$(e^2,e^3)$–plane, while $e^1$ is kept fixed.  This frame is related
by a rigid spatial rotation to the tilted--axis picture of
Appendix~\ref{app:tetrad} and is physically equivalent.

Let $\tilde e^a$ denote the aligned tetrad of
Eqs.~\eqref{eq:ref-tetrad-0}--\eqref{eq:ref-tetrad-3} in the main
text.  The precessing tetrad is
\begin{equation}
e^a = \Lambda^a{}_b(\varepsilon(t))\,\tilde e^b,
\qquad
\Lambda(\varepsilon)=
\begin{pmatrix}
1 & 0 & 0 & 0 \\
0 & 1 & 0 & 0 \\
0 & 0 & \cos\varepsilon & -\sin\varepsilon \\
0 & 0 & \sin\varepsilon & \cos\varepsilon
\end{pmatrix}.
\end{equation}
Expanding for small $\varepsilon$,
\begin{equation}
\cos\varepsilon\simeq 1-\frac{\varepsilon^2}{2},
\qquad
\sin\varepsilon\simeq \varepsilon,
\end{equation}
gives
\begin{align}
e^1 &= \tilde e^1,\\
e^2 &\simeq \tilde e^2 + \varepsilon\,\tilde e^3
        -\frac{\varepsilon^2}{2}\,\tilde e^2,\\
e^3 &\simeq \tilde e^3 - \varepsilon\,\tilde e^2
        -\frac{\varepsilon^2}{2}\,\tilde e^3.
\end{align}

\subsection{Torsion tensor to $\mathcal{O}(\varepsilon^2)$}

Using the Weitzenböck connection
$\Gamma^\lambda{}_{\mu\nu} = e_a{}^\lambda \partial_\nu e^a{}_\mu$,
the torsion tensor is
\begin{equation}
T^\lambda{}_{\mu\nu}
 = \Gamma^\lambda{}_{\nu\mu} - \Gamma^\lambda{}_{\mu\nu}.
\end{equation}
In an orthonormal frame this can be written as
$T^a = de^a + \omega^a{}_b\wedge e^b$; in the teleparallel gauge
$\omega^a{}_b=0$ so $T^a=de^a$.

Carrying out the expansion to second order in $\varepsilon$ and
keeping only the terms relevant for the precession mode, one finds
non–vanishing corrections such as
\begin{align}
\delta T^1{}_{\theta\phi} &= \varepsilon^2\,
  2q r_0 \sin\theta,\\
\delta T^2{}_{\psi\theta} &= \varepsilon^2\,
  m\omega r_0 \sin\theta,\\
\delta T^3{}_{\psi\theta} &= -\varepsilon^2\,
  m\omega r_0 \cos\theta,
\end{align}
together with similar terms related by symmetry.  All other components
either vanish or contribute only at higher orders in $\varepsilon$.

\subsection{Quadratic torsion scalar and symmetry protection}

The torsion scalar is
\begin{equation}
\mathbb{T}
 = \frac{1}{4}T^{\rho\mu\nu}T_{\rho\mu\nu}
 + \frac{1}{2}T^{\rho\mu\nu}T_{\nu\mu\rho}
 - T_\rho T^\rho,
\qquad
T_\rho = T^\mu{}_{\rho\mu}.
\end{equation}
Inserting the expanded tetrad and keeping only the
$\mathcal{O}(\varepsilon^2)$ terms, one can schematically write
\begin{equation}
\mathbb{T}^{(2)}
 = \varepsilon^2\Bigl[m^2\omega^2 r_0^2
   + (\text{terms}\propto q^2,\; q m\omega,\;\dots)\Bigr].
\end{equation}

A crucial point, confirmed both analytically and in the SymPy
implementation, is that \emph{all terms proportional to
$q^2\varepsilon^2$ cancel identically}.  This cancellation is
protected by the spherical symmetry of the monopole background:
a rigid rotation of a spherically symmetric configuration cannot
generate a restoring force proportional to $q^2$.  What survives at
$\mathcal{O}(\varepsilon^2)$ is the interference between the
anisotropic twist ($m$) and spin ($\omega$) sectors.

After the cancellation we are left with
\begin{equation}
\mathbb{T}^{(2)} = \varepsilon^2\,m^2\omega^2 r_0^2.
\end{equation}

\subsection{Integration and effective stiffness}

Integrating over the two–sphere and along the handle we obtain
\begin{align}
\int d^4x\, e\,\mathbb{T}^{(2)}
 &= \varepsilon^2 m^2\omega^2 r_0^2 L_\psi
    \int_0^\pi\sin\theta\,d\theta
    \int_0^{2\pi} d\phi \nonumber\\
 &= \varepsilon^2 m^2\omega^2 r_0^2 L_\psi
    \times 4\pi.
\end{align}
Up to an overall numerical factor that depends on conventions and
normalisation of the tetrad, this contribution induces an effective
action
\begin{equation}
S_{\rm TEGR}^{(2)}
 = \frac{1}{2}\int dt\,k(q)\,\varepsilon(t)^2,
\qquad
k(q) = \kappa_k\,L_\psi m^2\omega^2 r_0^2,
\end{equation}
with $\kappa_k=\mathcal{O}(1)$.  Using $r_0^2\propto |q|$ from
Appendix~\ref{app:phase1} we obtain the scaling
\begin{equation}
k(q)\propto \omega^2 |q|
\qquad\Rightarrow\qquad
\gamma = 1
\end{equation}
in the notation of the main text.

In the compact SymPy implementation described in the supplementary
material the same calculation yields a specific numerical value
$\kappa_k = 8\pi/9$ in our units.  This difference with respect to
the analytic coefficient $4\pi$ obtained above originates from a
slightly different normalisation of the frame fields and does not
affect the $|q|$– and $\omega$–scaling.  Since the main text never
uses the absolute value of $\kappa_k$, we keep it symbolic here.

\begin{lstlisting}[caption={SymPy code fragment used for the
  precession stiffness calculation.  The full listing is provided in
  the supplementary material.},label={code:appe}]
# Python/SymPy code fragment – full listing in supplementary material
e2 = cos(eps)*e2_tilde + sin(eps)*e3_tilde
e3 = -sin(eps)*e2_tilde + cos(eps)*e3_tilde
T2 = de2  # teleparallel gauge: T^a = de^a
T3 = de3
T2_quad = T2.subs(eps, 0).series(eps, 0, 3).coeff(eps**2)
# → shows q**2 terms cancel; only m*omega terms survive
\end{lstlisting}



%%%%%%%%%%%%%%%%%%%%%%%%%%%%%%%%%%%%%%%%%%%%%%%%%%%%%%%%%%%%%%%%%%%%%%%%%%%%%%
\section{Universality of the $\gamma=1$ Scaling}
\label{app:universality}
%%%%%%%%%%%%%%%%%%%%%%%%%%%%%%%%%%%%%%%%%%%%%%%%%%%%%%%%%%%%%%%%%%%%%%%%%%%%%%

The result $k(q)\propto \omega^2|q|$ obtained above might at first
sight appear to be an artefact of the specific $SU(2)$ parametrisation
used for the microscopic handle.  In this appendix we argue that,
within the class of axially symmetric spinning/twisted handles
considered in this paper, the scaling with $\gamma=1$ is in fact
generic.

The static energy is controlled by two competing contributions:
\begin{itemize}
\item a ``core'' term associated with the torsional monopole flux,
      scaling as $V_{\rm core}\sim q^2/r_0^2$;
\item a ``twist'' term associated with gradients along the fibre,
      scaling as $V_{\rm twist}\sim m^2 r_0^2$.
\end{itemize}
These depend only on the topological charges $(q,m)$ and on the
radius $r_0$, but not on the detailed choice of tetrad within a
given symmetry class.  Minimising
$V_{\rm core}+V_{\rm twist}$ therefore always gives
$r_0^2\propto |q|/|m|$ up to an $\mathcal{O}(1)$ factor, independently
of the microscopic parametrisation.

The precession stiffness $k(q)$ originates from the \emph{dynamical}
response of the twisted/spinning configuration to a small tilt.
In any axially symmetric handle with monopole flux $q$, twist $m$
and spin $\omega$, the torsion scalar contains terms of the schematic
form
\begin{equation}
\mathbb{T} \supset \omega m\,(\text{spatial rotation terms}),
\end{equation}
whose quadratic response to a uniform tilt produces an energy density
proportional to $m^2\omega^2$.
Integrating this density over the cross–section of area
$\sim r_0^2$ and along the handle then yields
\begin{equation}
k(q)\propto \omega^2 m^2 r_0^2
          \propto \omega^2 |q|.
\end{equation}
Thus, within this axially symmetric class, the exponent
$\gamma=1$ is a robust consequence of (i) the $q^2/r_0^2$ vs.
$m^2 r_0^2$ competition that fixes $r_0^2\propto |q|$, and
(ii) the fact that the precession mode couples to the twist and spin,
not directly to the monopole flux.

We do \emph{not} claim that $\gamma=1$ holds as a rigorous theorem
for arbitrary teleparallel configurations.  Extending the analysis
to more general, non–axially symmetric handles is left as an open
problem for future work.



%%%%%%%%%%%%%%%%%%%%%%%%%%%%%%%%%%%%%%%%%%%%%%%%%%%%%%%%%%%%%%%%%%%%%%%%%%%%%%
\section{The Static Case ($\omega=0$): Phenomenological Sterility}
\label{app:static}
%%%%%%%%%%%%%%%%%%%%%%%%%%%%%%%%%%%%%%%%%%%%%%%%%%%%%%%%%%%%%%%%%%%%%%%%%%%%%%

The spinning ansatz used in the main text is technically more
involved than the purely static handle, but it is precisely the
background spin $\omega\neq 0$ that makes the microscopic handle
phenomenologically interesting.  In this appendix we briefly
summarise what happens when the spin is switched off.

\subsection{Vanishing linear term in the effective potential}

The Nieh--Yan contribution to the effective action for a single
handle is schematically
\begin{equation}
S_{\rm NY}
 \propto \theta_{\rm NY}\,q\,\omega m\,\Delta r^2,
\end{equation}
with $\Delta r^2$ the difference in $r^2$ between the two ends of
the handle (see Appendix~\ref{app:phase2}).  When $\omega=0$ this
term vanishes exactly, regardless of the value of $\Delta r^2$.
The effective potential $V_{\rm eff}(\varepsilon;q)$ therefore has
no linear term in $\varepsilon$ and remains parity even, yielding
no geometric mechanism for chirality selection.

\subsection{Vanishing topological coupling to statistics}

More fundamentally, the Nieh--Yan density takes the form
\begin{equation}
\mathcal{N}
 \propto q\,\omega m\,
  d(r^2)\wedge dt\wedge\text{(angular 2--form)}.
\end{equation}
When $\omega=0$ the $dt$ component disappears and the spacetime
integral of $\mathcal{N}$ reduces to a trivial surface term that
does not generate a non–zero topological coupling.
In particular, the heuristic link between the Nieh--Yan invariant
and a Wess--Zumino--Witten term controlling particle statistics
(used as motivation in the main text) is absent in the static case,
so the handle does not provide a natural geometric origin for
half–integer spin states.

\subsection{Mechanical stability vs.\ phenomenological sterility}

The configuration with $\omega=0$ remains mechanically stable:
the static balance between $V_{\rm core}$ and $V_{\rm twist}$ is
unchanged, and the stiffness of small geometric deformations is
still set by $k(q)\propto m^2 r_0^2 L_\psi$ (which is non--zero
independently of $\omega$, see Appendix~\ref{app:phase3-calc}).
In this sense the static handle is not catastrophically unstable.

However, precisely the phenomena that motivated our construction in
the first place --- chirality selection, a possible link to
fermionic statistics, and near–critical binding of composite states
--- all rely on the interplay between spin $\omega$ and the Nieh--Yan
boundary term.  When $\omega=0$ this interplay is absent, and the
handle becomes phenomenologically inert.

The non–zero background spin $\omega$ should therefore be viewed not
as a mere technical decoration of the ansatz, but as the crucial
ingredient that ``breathes phenomenological life'' into the
microscopic handle.


%-----------------------------
% References
%-----------------------------

\bibliographystyle{unsrt}
\bibliography{refs}

\end{document}
